% $Log: abstract.tex,v $
% Revision 1.1  93/05/14  14:56:25  starflt
% Initial revision
%
% Revision 1.1  90/05/04  10:41:01  lwvanels
% Initial revision
%
%
%% The text of your abstract and nothing else (other than comments) goes here.
%% It will be single-spaced and the rest of the text that is supposed to go on
%% the abstract page will be generated by the abstractpage environment.  This
%% file should be \input (not \include 'd) from cover.tex.
Quantification of tissue perfusion from dynamic contrast-enhanced ultrasound data relies on appropriate modeling of the curve representing the evolution of the contrast-agent concentration inside the studied tissue.
Many factors, experimental or physiological, make inter-subject or intra-subject comparison of these perfusion parameters difficult.
In this thesis, the reproducibility and the comparison of various quantification methods was investigated through preclinical test-retest experiments and through simulations.
The investigated methods were: the log-normal model, the one-compartment model using an arterial input function, and the one-compartment model using a reference tissue.
The preclinical experiments revealed the difficulty to estimate an arterial input function directly from the image, as well as the necessity to locally correct for the time of arrival of the contrast agent in the tissue in order to ensure the model accurately fits the experimental enhancement curves.
A regularized linear estimation of the parameters of the one-compartment model using a reference tissue taking advantage of multiple tissue regions was then proposed to obtain homogeneous relative values of the tissue blood flow and tissue blood volume, expressed relatively to the parameter value inside the reference tissue.
The improved robustness and reproducibility of the method was demonstrated.
The influence of factors such as the exam duration, the sampling frequency, the number of tissue regions in the analysis, and the noise amplitude were investigated through simulations, and allowed us to formulate recommendations regarding the acquisition and the analysis of contrast-enhanced ultrasound studies.
