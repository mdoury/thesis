% $Log: abstract.tex,v $
% Revision 1.1  93/05/14  14:56:25  starflt
% Initial revision
%
% Revision 1.1  90/05/04  10:41:01  lwvanels
% Initial revision
%
%
%% The text of your abstract and nothing else (other than comments) goes here.
%% It will be single-spaced and the rest of the text that is supposed to go on
%% the abstract page will be generated by the abstractpage environment.  This
%% file should be \input (not \include 'd) from cover.tex.
La quantification de la perfusion tissulaire \`a partir de donn\'ees dynamiques d'\'echographie de contraste repose sur une mod\'elisation appropri\'ee de la cin\'etique de la concentration en agent de contraste dans le tissu \'etudi\'e. De nombreux facteurs, exp\'erimentaux ou physiologiques, rendent la comparaison de ces param\`etres de perfusion difficile.
Dans cette th\`ese, la reproductibilit\'e et la comparaison de diff\'erentes m\'ethodes de quantification ont \'et\'e \'etudi\'ees dans le cadre d'une \'etude pr\'eclinique (test-retest) et sur des simulations num\'eriques. Les m\'ethodes \'etudies ont \'et\'e : le mod\`ele log-normal, le mod\`ele compartimental avec fonction d'entr\'ee et le mod\`ele compartimental avec tissu de r\'ef\'erence. Les \'etudes pr\'ecliniques ont montr\'e la difficult\'e d'estimation d'une fonction d'entr\'ee art\'erielle et la n\'ecessit\'e de corriger localement le temps d'arriv\'ee de l'agent de contraste dans le tissu pour que l'approximation des cin\'etiques exp\'erimentales par le mod\`ele soit de qualit\'e suffisante.
Une estimation lin\'eaire sous contrainte des param\`etres du mod\`ele compartimental avec tissu de r\'ef\'erence a \'et\'e ensuite propos\'ee pour obtenir \`a l'\'echelle r\'egionale et/ou locale des valeurs relatives coh\'erentes de d\'ebit sanguin tissulaire et de volume sanguin tissulaire, exprim\'ees par rapport aux valeurs dans le tissu de r\'ef\'erence. Il a \'et\'e montr\'e que cette approche est la plus robuste et la plus reproductible. L'influence des facteurs tels que la dur\'ee d'acquisition, la fr\'equence d'\'echantillonnage, le nombre de r\'egions utilis\'ees et l'amplitude du bruit a \'et\'e \'etudi\'ee sur des simulations et a permis de formuler des recommandations pour l'acquisition et le traitement des \'etudes en \'echographie de contraste.
