%!TeX root = ./main.tex
%% This is an example first chapter.  You should put chapter/appendix that you
%% write into a separate file, and add a line \include{yourfilename} to
%% main.tex, where `yourfilename.tex' is the name of the chapter/appendix file.
%% You can process specific files by typing their names in at the 
%% \files=
%% prompt when you run the file main.tex through LaTeX.
% \cleardoublepage
\chapter{Conclusion}\label{chapter:conclusion}
In Chapter~\ref{chapter:review} we reviewed the semi-quantitative, deconvolution, and compartmental methods to quantify tissue perfusion in various imaging modalities.
The difficulties in the detection of the arterial input function yielded to the development of reference tissue models for relative quantification of perfusion.
Indeed, a reference tissue can be selected in a large, well perfused area of the image, alleviating the risks of partial volume effect, and saturation artifacts.
While reference tissue models were used to quantify perfusion in positron emission tomography, X-ray computed tomography, and magnetic resonance exams, they were never applied to contrast-enhanced ultrasound data.

In Chapter~\ref{chapter:PMB} we compared a semi-quantitative approach based on the log-normal model \textbf{LN} to a one-compartment model using an arterial input function (\textbf{AIF}) in terms of reproducibility through preclinical test-retest contrast-enhanced ultrasound experiments.
This study revealed the higher reproducibility of \textbf{AIF} model compared to the semi-quantitative parameters of the \textbf{LN} model.
But the study also revealed the difficulties encontered in the estimation of the arterial input function and the impact of these variations on the estimated perfusion parameters.
Normalizing perfusion parameters of the \textbf{LN} and \textbf{AIF} models according to a reference tissue improved inter-exam reproducibility.
The direct estimation of relative perfusion parameters using a one-compartment reference tissue (\textbf{RT}) yielded the most reproducible parameters in our test-retest study.

In Chapter~\ref{chapter:PMB2} we established the relations between the parameters of the \textbf{LN}, \textbf{AIF}, and \textbf{RT} models.
These relations reveal the strong link between the semi-quantitative parameters of the \textbf{LN} model and the parameters of the compartmental approaches, but also explain the inter-exam variations observed in semi-quantitative parameters.
Normalizing perfusion parameters according to a reference tissue results in an improved robustness to inter-exam variations.
These analytical considerations were also verified experimentally in preclinical test-retest data.

In Chapter~\ref{chapter:IUS} we proposed a new regularized linear estimation method (\textbf{rReg}) for the relative perfusion parameters of \textbf{RT} model.
The model takes advantage of the functional heterogeneity of the tumor to regularize the estimation according to the reference tissue.
The reproducibility of the regularized model was assessed on the same preclinical test-retest data as in our previous studies.
Regularization significantly improved the reproducibility of perfusion parameters, in particular the improved reproducibility of relative blood flow was reported.

In Chapter~\ref{chapter:IRBM} and Chapter~\ref{chapter:PLOSONE}, we assessed the accuracy and precision of the estimates of the \textbf{rReg} model through simulation experiments.
Chapter~\ref{chapter:IRBM} focuses on the impact of recirculating contrast agent on the estimation of perfusion parameters.
In Chapter~\ref{chapter:PLOSONE} we presented the \textbf{rReg} in more details, and studied the impact of varying data characteristics, e.g.~exam duration, sampling period, or of quantification choices, e.g.~reference tissue, number of tissue regions.
These studies revealed the robustness of the \textbf{rReg} model to recirculation and degraded data, but also its ability to quantify perfusion homogeneously across tumor regions.



% \newpage
% \bibliographystyle{plainnat}
% \bibliography{Bibliography/chap8}