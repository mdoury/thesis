%!TeX root = ./main.tex
%% This is an example first chapter.  You should put chapter/appendix that you
%% write into a separate file, and add a line \include{yourfilename} to
%% main.tex, where `yourfilename.tex' is the name of the chapter/appendix file.
%% You can process specific files by typing their names in at the 
%% \files=
%% prompt when you run the file main.tex through LaTeX.
% \cleardoublepage
\chapter{Conclusion}\label{chapter:conclusion}
In Chapter~\ref{chapter:review} we reviewed the semi-quantitative, deconvolution, and compartmental methods to quantify tissue perfusion in various imaging modalities.
Semi-quantitative are often used as perfusion indicators, especially in combination with indicator dilution theory, indeed they are easily derived from enhancement curve in the tissue of interest only.
Classical compartmental models require the knowledge of the arterial input function, which can either be obtained through blood sampling or from the image.
Blood sampling is extremely invasive, and image-based estimation suffers from various artifacts, e.g.~partial volume, saturation.
The difficulties in the detection of the arterial input function yielded to the development of reference tissue models for relative quantification of perfusion.
Indeed, a reference tissue can be selected in a large, well perfused area of the image, alleviating the risks of partial volume effect, and saturation artifacts.
While reference tissue models were used to quantify perfusion in positron emission tomography, X-ray computed tomography, and magnetic resonance exams, they were never applied to contrast-enhanced ultrasound data.

In Chapter~\ref{chapter:PMB} we compared a semi-quantitative approach based on the log-normal model \textbf{LN} to a one-compartment model using an arterial input function (\textbf{AIF}) in terms of reproducibility through preclinical test-retest contrast-enhanced ultrasound experiments.
This study revealed the higher reproducibility of \textbf{AIF} model compared to the semi-quantitative parameters of the \textbf{LN} model.
But the study also revealed the difficulties encontered in the estimation of the arterial input function and the impact of these variations on the estimated perfusion parameters.
Normalizing perfusion parameters of the \textbf{LN} and \textbf{AIF} models according to a reference tissue improved inter-exam reproducibility.
The direct estimation of relative perfusion parameters using a one-compartment reference tissue (\textbf{RT}) yielded the most reproducible parameters in our test-retest study.

In Chapter~\ref{chapter:PMB2} we established the relations between the parameters of the \textbf{LN}, \textbf{AIF}, and \textbf{RT} models.
These relations reveal the strong link between the semi-quantitative parameters of the \textbf{LN} model and the parameters of the compartmental approaches, but also explain the inter-exam variations observed in semi-quantitative parameters.
Normalizing perfusion parameters according to a reference tissue results in an improved robustness to inter-exam variations.
These analytical considerations were also verified experimentally in preclinical test-retest data.

In Chapter~\ref{chapter:IUS} we presented a linear formulation of the \textbf{RT} model and revealed its limitations when considering multiple tissues of interest or multiple regions in a single tissue. 
We proposed a new regularized linear estimation method (\textbf{rReg}) for the relative perfusion parameters of \textbf{RT} model, and compared it to the unregularized linear estimation method (\textbf{rLin}).
The \textbf{rReg} model takes advantage of the functional heterogeneity of the tumor to regularize the estimation according to the reference tissue.
The reproducibility of the two models was assessed on the same preclinical test-retest data as in our previous studies.
Regularization significantly improved the reproducibility of perfusion parameters, in particular the reproducibility of relative blood flow was found significantly better.

In Chapter~\ref{chapter:IRBM} and Chapter~\ref{chapter:PLOSONE}, we assessed the accuracy and precision of the estimates of the \textbf{rReg} model through simulation experiments.
Chapter~\ref{chapter:IRBM} focuses on the impact of recirculating contrast agent on the estimated perfusion parameters.
In Chapter~\ref{chapter:PLOSONE} we presented the \textbf{rReg} in more details, and studied the impact of varying data characteristics, e.g.~exam duration, sampling period, or of quantification choices, e.g.~reference tissue, number of tissue regions.
These studies revealed the increased robustness of the \textbf{rReg} model to recirculation and to degraded data, but also its ability to quantify perfusion homogeneously across tumor regions, allowing meaningful intra-exam parameter comparison.

Overall, the \textbf{rReg} model proved a promising perfusion quantification tool, and its applicability to contrast-enhanced ultrasound data was demonstrated in this thesis.
When performing relative perfusion quantification in multiple regions, we highly recommend accounting for the relations between the regional model parameters to avoid regional inconsistencies, while making the estimation more robust.
The model could be applied to reveal tumor functional heterogeneity at a finer scale, considering pixels or macro-pixels as regions for instance.
The lack of pixel-to-pixel correspondance in our test-retest experiments motivated the regional cutout used in this thesis to ensure meaningful inter-exam parameter comparison, indeed the overlap between corresponding regions generally increased with their area.

Two-dimensional data still represents the majority of contrast-enahced ultrasound exams nowadays, making exam comparison difficult.
Indeed it is particularly difficult to ensure the exact same imaging plane is selected in two exams performed days apart, and this is especially true when monitoring evolving tissues where structures change in shapes and size, e.g.~tumor growth, treatment response.
Therefore we highly recommend the use three-dimensional perfusion imaging whenever possible, in particular when it comes to exam comparison.
The recent performance improvements of three-dimensional ultrasound scanners is therefore promising for tumor monitoring.
Indeed ultrasound imaging will be able to catch up on tomographic imaging modalities, giving access to a more relevant information regarding the shape, size, structure, and function of the lesions, while retaining its real-time, non-ionizing, and cost-effective characteristics.

Three-dimensional perfusion imaging is vastly available in positron emission tomography, X-ray computed tomography, and magnetic resonance imaging.
Applicability of the \textbf{rReg} model to other perfusion imaging modalities should be further investigated, and in some cases may require some adaptations to account for the characteristics of the data, the tissue, or the contrast agent.
Indeed, adapting the estimation method to various compartmental architectures should be investigated in order to account for the characteristics of the tissue and contrast agent.
Further adaptation would also be necessary in order to study perfusion in the liver, an appropriate model would account for both the arterial and portal blood supplies of the tissue.
Similarly, the vascular network of the kidney is complex and results in multiple perfusion phases.
Using the kidney as a reference tissue while incorporating the multiple phases in the model may further improve the quality of the estimation.

The impact of the reference tissue on the perfusion parameters of the \textbf{rReg} was shown in Chapter~\ref{chapter:PLOSONE}, however a finer study is necessary to better define the characteristics of the ideal reference tissue and characterize the impact of an unideal reference tissue. 
Accounting for multiple reference tissues may alleviate the sensitivity of the model estimates to the characteristics of the reference tissues. 
For instance, \citet{Yang:2007ki} used multiple reference tissues to estimate the arterial input function in contrast-enhanced magnetic resonance exams.

To conclude, choosing the quantification method to assess tissue perfusion through contrast-enhanced exams remains a complex problem that heavily depends on the data and on the goal of the study.
Exam comparison is particularly difficult because of variations occuring between exams at both the experimental and physiological levels.
This thesis demonstrated the ability of relative one-compartment models to quantify contrast-enhanced ultrasound exams reproducibly and robustly at a regional scale, revealing the functional heterogeneity of tumors while enabling meaningful intra-exam and inter-exam comparison.
We emphasis the necessity to consider the relations between the perfusion parameters of various tissues, or tissue regions, as it proved to considerably improve the accuracy, robustness, and reproducibility of perfusion parameters in the \textbf{rReg} model.
However, using reference tissues to monitor the perfusion of tumors undergoing therapy raises a question regarding the effect of the treatment on the reference tissue, especially in the case of anti-angiogenic treatments.
Reference tissue models proved more reproducible and robust than arterial input function models overall, however once accurate image-based estimation of the arterial input function is achievable, will it still be the case?


\newpage
\bibliographystyle{plainnat}
\bibliography{Bibliography/chap8}