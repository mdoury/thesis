\begin{otherlanguage}{francais}
\subsection*{Introduction}
Cette th\`ese effectu\'e au sein du {\em Laboratoire d'Imagerie Biom\'edicale} (LIB) a \'et\'e financ\'e par la {\em Fondation pour la Recherche M\'edicale} (FRM).
Le projet global consiste \`a d\'evelopper un outil de classification multi-param\'etrique des tissus tumoraux exploitant diverses modalit\'es d'imagerie ultrasonore, i.e.~\'echographie quantitative, elastosonographie et \'echographie de contraste.
Les donn\'es serviront au d\'eveloppement d'un mod\`ele r\'ealiste de croissance tumorale ainsi que de la r\'eponse aux traitements anti-tumoraux.
La premi\`ere \'etape de ce projet consistait donc \`a estimer de fa\c{c}on pr\'ecis\'ement et reproductible des param\`etres de perfusion \`a partir de donn\'es de contrast ultrasonores, ce afin de les utiliser dans un contexte de suivi longitudinal.

La quantification de la perfusion est une t\^ache difficile, en effet des variations peuvent intervenir entre les examens, que ce soit au niveau exp\'erimental ou physiologique. 
Cependant ce processus s'av\`ere crucial lorsque l'on cherche \`a \'etudier la croissance de tumeurs, avec ou sans traitement.
L'imagerie de contraste permet d'\'etudier la perfusion in-vivo, cependant la comparaison quantitative d'examens reste difficile en raison du manque de reproductibilit\'e des acquisitions.
La majorit\'e des m\'ethodes de quantification de la perfusion ont \'et\'e d\'evelopp\'ees pour analyser des donn\'ees \`a une \'echelle globale, ce qui soit masque les variations spatiale de la perfusion tissulaire, soit n\'eglige les relations entre les param\`etres locaux.
Le but de cette th\`ese est de rendre l'estimation de param\`etres de perfusion robustes aux variations inter-examens, afin de rendre possible la comparaison d'examens tout en r\'ev\'elant l'h\'et\'erog\'en\'eit\'e spatiale de la fonction vasculaire.
Notre \'etude se concentre sur la quantification de l'\'echographie de contraste, cependant l'usage des m\'ethodes propos\'ees pour la quantification d'autres modalit\'es d'imagerie de contraste pourrait-\^etre \'etudi\'ee.

Le manuscrit est divis\'e en trois parties.
La premi\`ere partie cherche \`a \'etablir un \'etat de l'art des m\'ethodes de quantification de la perfusion d\'evelopp\'ees pour les diff\'erentes modalit\'es d'imagerie de contraste.
La seconde partie \'etudie la reproductibilit\'e des param\`etres obtenus \`a l'aide de diff\'erentes approches ainsi que les relations qui les unissent, i.e.~une approche semi-quantitative, un mod\`ele \`a un compartiment utilisant une fonction d'entr\'e art\'erielle, et un mod\`le \`a un compartiment utilisant un tissu de r\'ef\'erence.
Enfin, dans la troisi\`eme partie nous pr\'esentons une nouvelle approche d'estimation du mod\`ele \`a un compartiment utilisant un tissu de r\'ef\'erence exploitant et r\'ev\'elant l'h\'et\'erog\'en\'eit\'e spatiale de la fonction vasculaire.

\subsection*{Partie I.~Quantification de la perfusion: \'etat de l'art}
Dans cette premi\`ere partie, compos\'ee uniquement du Chapitre~\ref{chapter:review} nous \'etablissons un \'etat de l'art des m\'ethodes d\'evelopp\'ees pour quantifier la perfusion tumorale.
Ce Chapitre pr\'esente une s\'election de m\'ethodes de quantification fondatrices, ainsi que leur \'evolution \`a travers les diff\'erentes modifications qui y ont \'et\'e aport\'ees pour prendre en compte diff\'erents facteurs. 

Les m\'ethodes de quantification sont class\'ees en trois cat\'egories : semi-quantitatives, d\'econvolution, ou compartimentales.
Les approches semi-quantitatives extraient des param\'etres caract\'erisant la cin\'etique de la concentration en agent de contraste et sont courrament utilis\'es pour caract\'eriser la perfusion tissulaire, mais les param\`tres de ces mod\`les n'ont pas de lien direct avec la physiologie.
Les approches de d\'econvolution, ainsi que la majorit\'e des approches compartimentales, n\'ecessitent la connaissance de la fonction d'entr\'ee art\'erielle.
La fonction d'entr\'ee art\'erielle peut-\^etre obtenue par pr\'el\`evement sanguin, ou directement dans l'image. 
Cependant, les pr\'el\`evements sanguins sont invasifs, en particulier les pr\'el\`evements art\'eriels, et en raison des fortes concentrations en agent de contraste observ\'ees dans les art\`eres l'\'estimation directe de la fonction d'entr\'e art\'erielle dans l'image est affect\'ee par des artefacts, y compris des effets de saturation et de volumes partiels.

Les difficult\'es rencontr\'ees lors de l'estimation de la fonction d'entr\'ee art\'erielle ont motiv\'e le d\'eveloppement de m\'ethodes utilisant un tissu de r\'ef\'erence, permettant la quantification relative de la perfusion.
En effet, un tissu de r\'ef\'erence peut \^etre choisi dans une r\'egion \'etendue et bien perfus\'e de l'image, r\'eduisant ainsi les risques de saturation et de volumes partiels.
\`A notre connaissance, es mod\`eles compartimentaux, qu'ils utilisent une fonction d'entr\'e art\'erielle ou un tissu de r\'ef\'erence, ont \'et\'e appliqu\'es \`a la quantifiction de la perfusion dans la plupart des modalit\'es d'imagerie de contraste, \`a l'exception de l'\'echographie.

\subsection*{Partie II.~Reproductibilit\'e des m\'ethodes existantes et les relations entre les diff\'erentes approches}
Cette partie est principalement compos\'ee du Chapitre~\ref{chapter:PMB} dans lequel nous nous interrogeons sur la reproductibilit\'e des param\`etres de perfusion estim\'ees par diff\'erentes m\'ethodes de quantification, mais \'egalement du Chapitre~\ref{chapter:PMB2} qui compl\'emente le Chapitre pr\'ec\'edent en \'etablissant les relations th\'eoriques et empiriques entre les param\`etres des diff\'erents mod\`eles, et justifiant le choix des parm\`etres s\'electionn\'es dans l'\'etude de reproductibilit\'e.

L'\'etude pr\'esent\'ee dans le Chapitre~\ref{chapter:PMB} a fait l'objet d'une publication dans {\em Physics and Medicine in Biology} sous le titre \``Quantification of tumor perfusion using dynamic contrast-enhanced ultrasound: impact of mathematical modeling\''.
Nous y pr\'esentons une \'etude de reproductibilit\'e des param\`etres de perfusion en \'echographie de contraste portant sur quatre souris porteuses de tumeurs colorectales.
Chaque souris a subit quatre examens test-retest d'\'echographie de contraste avec injection automatique d'un bolus de microbulles suivant un protocole d\'evelopp\'e au laboratoire.
Les acquisitions \'etaient cependant espac\'ees de quinze minutes afin de garantir l'\'elimination totale des microbulles avant un nouvel examen.
La surface perfus\'ee des tumeurs a ensuite \'et\'e segment\'ee en 32 r\'egions, permettant la d\'efinition de 32 cin\'etiques r\'egionales.
Ce d\'ecoupage r\'egional a \'et\'e choisi afin de r\'ev\'eler l'h\'et\'erog\'en\'eit\'e de la perfusion tumorale tout en garantissant un rapport signal sur bruit permettant la quantification et en permettant la comparaison des param\`etres d'un examen.
Enfin, diverses m\'ethodes de quantification ont \'et\'e appliqu\'ees \`a ces cin\'etiques r\'egionales afin d'estimer des param\`etres de perfusion.

En particulier, nous avons compar\'e la reproductibilit\'e des param\`etres de perfusion estim\'es \`a l'aide de (a)~une approche bas\'ee sur le mod\`ele log-normal (\textbf{LN}) estimant des param\`etres semi-quantitatifs, e.g.~l'aire sous la courbe $AUC$ et le taux de remplissage $WIR$, (b)~un mod\`ele \`a un compartiment utilisant une fonction d'entr\'ee art\'erielle (\textbf{AIF}) estimant des param\`tres absolus, i.e.~le volume sanguin tissulaire $V$, le flux sanguin tissulaire $F$, et la constant de taux $k_T = \frac{F}{V}$, et (c)~un mod\`ele \`a un compartiment utilisant un tissu de r\'ef\'erence (\textbf{RT}) estimant des param\`tres relatifs, i.e.~le volume sanguin tissulaire relatif $rV_{RT}$, le flux sanguin tissulaire relatif $rF_{RT}$, la constante de taux dans le tissu d'int\'er\^et $k_{T}$, et la constant de taux dans le tissu de r\'ef\'erence $k_R = \frac{rV \cdot k_T}{rF}$.
Dans cette \'etude, les trois mod\`eles sont ajust\'es aux cin\'etiques de perfusion exp\'erimentales en minimisant l'erreur au sens des moindres carr\'es \`a l'aide d'algorithmes de minimisation non-lin\'eaires, et la valeur du param\`etre $k_R$ a \'et\'e fix\'ee lors de l'estimation du param\`etre

Afin d'\'etudier l'effet de la normalisation sur la reproductibilit\'e des param\`etres de perfusion, nous avons d\'efinit des param\`etres relatifs \`a partir des param\`etres du mod\`ele \textbf{LN}, not\'es $rAUC$ et $rWIR$, et des param\`etres absolus du mod\`ele \textbf{AIF}, not\'es $rV_{AIF}$ et $rF_{AIF}$, en les normalisant par leur valeur dans un tissu de r\'ef\'erence.
Enfin, la prise en compte du temps d'arriv\'ee du bolus dans le tissu \'etudi\'e \'etant souvent n\'eglig\'ee, nous avons \'etudi\'e son impact sur la qualit\'e de mod\'elisation des donn\'ees de perfusion.

Les param\`etres absolus du mod\`ele \textbf{LN} se sont montr\'e peu reproductibles avec des coefficients de variation m\'edians de l'ordre de 30\% pour $AUC$ et 40\% pour $WIR$, sugg\'erant une forte sensibilit\'e des param\`etres aux variations inter-examens.
Les param\`etres absolus du mod\`le \textbf{AIF} se sont r\'ev\'el\'es plus reproductible avec des coefficients de variation m\'edians de l'ordre de 25\% pour $V$ et 35\% pour $F$.
Cependant l'\'etude d\'emontre la sensibilit\'e de ces param\`etres \`a la fonction d'entr\'ee art\'erielle utilis\'e, ainsi que les difficult\'es rencontr\'ees lors de son estimation dans l'image en raison de la petite taille et des fortes intensit\'es caract\'eristiques.
La normalisation des param\`etres des mod\`eles \textbf{LN} et \textbf{AIF}, donnant ainsi naissance aux mod\`eles \textbf{rLN} et \textbf{rAIF}, a permi une r\'eduction significative de la variabilit\'e inter-examen.
En effet, les param\`etres li\'es au volume sanguin tissulaire ,$rAUC$ et $rV_{AIF}$, ont des coefficients de variation m\'edians de l'ordre de 20\%, tandis que les param\`tres li\'es au flux sanguin tissulaire, $rWIR$ et $rF_{AIF}$, ont des coefficients de variation m\'edians l\'eg\`erement sup\'erieurs \`a 30\%.
L'estimation directe de param\`etres relatifs \`a l'aide du mod\`le \textbf{RT} s'est r\'ev\'el\'ee la plus reproductible de l'\'etude, avec des coefficients de variation m\'edians l\'eg\`erement inf\'erieurs \`a 20\% pour $rV_{RT}$ et \`a 30\% pour $rF_{RT}$.

La n\'ecessit\'e de prendre en compte le temps d'arriv\'ee du bolus dans le tissu \'etudi\'e a \'et\'e d\'emontr\'ee en \'etudiant la qualit\'e de la mod\'elisation obtenue avec les diff\'erents mod\`eles compartimentaux selon qu'un param\`etre de temps de retard soit estim\'e ou non. 
En effet, le nombre de r\'egions pr\'esentant une mod\'elisation de mauvaise qualit\'e chute de 212 pour le mod\`le $AIF$ et 56 pour le mod\`ele $RT$, sur un total de 512 r\'egions \'etudi\'ees dans notre \'etude test-retest, \`a seulement 19 pour les deux mod\`les.

L'impact de la fonction d'entr\'ee sur les valeurs des param\`etres absolus et relatifs du mod\`ele \textbf{AIF} a \'et\'e \'etudi\'e en comparant les valeurs obtenues pour un examen donn\'e, en utilisant deux fonctions d'entr\'ees estim\'ees dans les images avec diff\'erentes valeurs de seuils.
Les param\`etres absolus se sont r\'ev\'el\'e extr\^emement sensibles \`a la fonction d'entr\'ee art\'erielle, $V$ et $F$ variant du simple au double.
Les param\`etres relatifs du mod\`ele \textbf{rAIF} se sont montr\'e plus robustes aux variations de la fonction d'entr\'ee art\'erielle, en particulier $rV_{AIF}$ qui est pratiquement insensible \`a la fonction d'entr\`ee utilis\`ee dans notre \'etude.

Le Chapitre~\ref{chapter:PMB2} pr\'esente les relations entre les param\`etres des diff\'erents mod\`eles \'etudi\'es dans le Chapitre~\ref{chapter:PMB}, i.e.~\textbf{LN}, \textbf{rLN}, \textbf{AIF} et \textbf{RT}.
Les relations sont d'abord \'etablies th\'eoriquement en \'etablissant les d\'efinitions analytiques des param\`etres d'un mod\`ele en fonction des param\`etres d'un autre mod\`ele, puis empiriquement \`a travers une \'etude de corr\'elation.
En particulier, les param\`etres des mod\`eles \textbf{LN} et \textbf{rLN} ont \'et\'e exprim\'es en fonction des param\`etres du mod\`ele \textbf{AIF} pour trois fonctions d'entr\'ee diff\'erentes, i.e.~dans les deux cas id\'ealis\'es que sont l'entr\'ee de type Dirac et de type porte, mais \'egalement dans le cas g\'en\'eral.
Ce chapitre d\'emontre notamment les fortes relations existant entre le param\`etre $AUC$ et le param\`etre de volume sanguin tissulaire $V$, et entre le param\`etre $WIR$ et le param\`etre de flux sanguin tissulaire $F$, ce constat a motiv\'e notre s\'election de param\`etres du mod\`le \textbf{LN} dans le Chapitre~\ref{chapter:PMB}. 
Mais il montre \'egalement la sensibilit\'e des param\`etres semi-quantitatifs \`a la fonction d'entr\'ee art\'erielle, et confirme l'int\'ere\^et de la normalisation pour s'absoudre de cette d\'ependance.
L'int\'er\^et de la normalisation est \'egalement confirm\'e empiriquement, en effet la corr\'elation entre les param\`etres relatifs des diff\'erents mod\`eles se trouve renforc\'ee.

\subsection*{Partie III.~Proposition et \'evaluation d'une nouvelle m\'ethode de quantification}
Dans cette troisi\`eme partie, nous pr\'esentons d'abord une approche d'estimation reposant sur une formulation lin\'eaire du mod\`ele \textbf{RT} adapt\'ee d'autres modalit\'es d'imagerie de contrast, not\'ee \textbf{rLin}, et permettant l'estimation d'un param\`etre suppl\'ementaire, i.e.~la constant de taux du tissu de r\'ef\'erence $k_R$.
Constatant qu'une valeur diff\'erente du param\`etre $k_R$ est estim\'ee par r\'egion tumorale, alors qu'elles caract\'erisent toutes le m\^eme tissu de r\'ef\'erence, nous avons propos\'e une nouvelle m\'ethode d'estimation r\'egularis\'ee, not\'ee \textbf{rReg}, exploitant l'h\'et\'erog\'en\'eit\'e spatiale de la perfusion en s'assurant qu'une valeur unique du param\`etre $k_R$ soit estim\'ee pour toutes les r\'egions d'un m\^eme examen.
Nous \'etudierons par la suite la robustesse de ces deux mod\`eles aux variations inter-examens, \`a des ph\'enom\`enes physiologiques comme la recirculation, \`a des param\`etres d'acquisition, ou encore \`a diff\'erentes strat\'egies dans l'analyse des donn\'ees.

Les \'equations des mod\`eles \textbf{rLin} et \textbf{rReg} sont d'abord d\'evelopp\'ees bri\`evement dans le Chapitre~\ref{chapter:IUS}.
En effet, le Chapitre~\ref{chapter:IUS} qui \'etudie la reproductibilit\'e des param\`etres de perfusion estim\'es avec les deux approches a fait l'objet d'une publication courte suite \`a la pr\'esentation de ces travaux lors du congr\`es {\em IEEE International Symposium on Biomedical Imaging} (ISBI) qui s'est tenu \`a Prague en avril 2016.
Le Chapitre~\ref{chapter:IRBM} \'etudie l'impact de la recirculation des microbulles sur la justesse et la pr\'ecistion des estimations, en utilisant des cin\'etiques de perfusion simul\'ees de fa\c{c}on r\'ealiste  \`a l'aide du mod\`ele \textbf{AIF}.
Cette \'etude a fait l'objet d'une publication dans la revue {\em Innovation and Research in BioMedical Engineering} (IRBM).
Dans le Chapitre~\ref{chapter:PLOSONE} nous revenons sur les d\'eveloppements th\'eoriques qui ont amen\'e le d\'eveloppement du mod\`ele \textbf{rReg}, puis nous nous proposons d'\'etudier l'impact de divers param\`etres sur la justesse et la pr\'ecision des estimations.
Nous nous somme notamment int\'eress\'e aux param\`etres d'acquisition comme la dur\'ee des examens ou la fr\'equence d'\'echantillonage, mais aussi \`a des strat\'egies d'analyse comme le nombre de r\'egions d'int\'er\^et ou le choix du tissu de r\'ef\'erence.
Nous pr\'evoyons de soumettre une version revue et all\'eg\'ee de ce Chapitre pour publication dans la revue {\em Medical Image Analysis}.

Dans le Chapitre~\ref{chapter:IUS}, les param\`etres de perfusion estim\'es \`a l'aide des m\'ethodes \textbf{rLin} et \textbf{rReg} sont compar\'es, en terme de reproductibilit\'e, aux param\`etres des mod\`ele \textbf{LN} et \textbf{rLN}, en se basant la m\^eme \'etude test-retest que dans le Chapitre~\ref{chapter:PMB}.
Le mod\`ele \textbf{rLin} s'est r\'ev\'el\'ee \^etre l'approche relative la moins robuste aux variations inter-examen, notamment en terme de param\`etre de flux puisque nous avons obtenu un coefficient de variation m\'edian sup\'erieur \`a 40\% pour $rF_{rLin}$.
Cette forte variabilit\'e montre la faible identifiabilit\'e du param\`etre $k_R$ et son impact sur les autres param\`etres du mod\`ele, notamment sur $rF_{rLin}$.
Elle explique \'egalement la pratique courante qui consiste \`a ne pas estimer $k_R$ mais \`a utiliser une valeur fixe provenant de la lit\`erature ou d'exp\'eriences pr\'ealables pour stabiliser l'estimation des autres param\`etres, comme nous l'avons fait nous m\^eme avec le mod\`ele \textbf{RT} pr\'esent\'e dans le Chapitre~\ref{chapter:PMB}.
L'approche r\'egularis\'e \textbf{rReg} s'est montr\'ee la plus reproductible dans cette \'etude, puisque nous avons obtenu des coefficients de variation m\'edians inf\'erieurs \`a 20\% pour $rV_{rReg}$, et inf\'erieurs \`a 30\% pour $rF_{rReg}$, dans notre \'etude test-retest.

Dans le Chapitre~\ref{chapter:IRBM}, nous avons \'etudi\'e la sensibilit\'e \`a la recirculation des param\`etres estim\'es \`a l'aide des mod\`eles \textbf{LN}, \textbf{rLN}, \textbf{rLin} et \textbf{rReg}.
En effet la recirculation de l'agent de contraste dans le tissu \'etudi\'e est un probl\`eme courrament rencontr\'e lorsque l'on cherche \`a quantifier la perfusion in vivo.
Pour s'absoudre de ce ph\'enom\`ene affectant l'allure des cin\'etiques de contraste, une approche simpliste consiste \`a \``couper\'' les donn\'ees avant que la recirculation n'intervienne.
Nous avons donc \'egalement \'etudi\'e la capacit\'e de cette technique \`a estimer des param\`etres de perfusion de fa\c{c}on juste et pr\'ecise en l'appliquant aux mod\`eles \textbf{LN} et \textbf{rLN}, dont les variantes sont not\'ees \textbf{LN$_{WI}$} et \textbf{rLN$_{WI}$}.

Les param\`etres absolus du mod\`ele \textbf{LN} se sont r\'ev\'el\'es particuli\`erement sensibles \`a la recirculation en raison de leur forte d\'ependance \`a la fonction d'entr\'ee art\'erielle.
En ce qui concerne le mod\`ele relatif \textbf{rLN}, l'estimation du param\`etre $rAUC$ s'est montr\'ee robuste \`a la recirculation, cependant le param\`etre $rWIR$ reste sensible \`a la recirculation.
Les approches \textbf{LN$_{WI}$} et \textbf{rLN$_{WI}$} se sont r\'ev\'el\'ees peu robustes, en effet malgr\'e l'accord entre les param\`etres estim\'es avec et sans recirculation, le nombre r\'eduit d'\'echantillons utilis\'es durant l'ajustement du mod\`ele ne permet pas d'ajuster le mod\`ele correctement \`a la phase descendante de la cin\'etique, et rend l'estimation particuli\`erement sensible au bruit.

Les param\`etres de flux et de volume sanguin tissulaires relatifs des mod\`eles \textbf{rLin} et \textbf{rReg} se sont r\'ev\'el\'es peu sensibles \`a la recirculation, avec un l\'eger avantage au mod\`ele \textbf{rLin}, ce qui confirme la robustesse structurelle des mod\`eles compartimentaux \`a cet \'egard.
En revanche, l'estimation des param\`etres $k_T$ et $k_R$ par le mod\`ele \textbf{rLin} est sujette \`a de fortes erreurs variant d'une r\'egion \`a l'autre, et ce m\^eme en l'absence de recirculation.
Le mod\`ele \textbf{rReg} n'est pas affect\'e par ce ph\'enom\`ene et montre une erreur d'estimation tr\`es faible pour ces deux param\`etres.
En outre, si les param\`etres $rV_{rReg}$ et $rF_{rReg}$ sont globalement plus affect\'es par la recirculation que les param\`etres $rV_{rLin}$ et $rF_{rLin}$, l'erreur est plus homog\`ene entre les diff\'erentes r\'egions d'int\'er\^et en utilisant le mod\`ele \textbf{rReg}, permettant ainsi une comparaison des param\`etres r\'egionaux plus robuste.

Dans le Chapitre~\ref{chapter:PLOSONE}, nous \'etudions l'impact de diff\'erents facteurs, pouvant \^etre la source d'erreurs de quantification, sur les param\`etres de perfusion des mod\`eles \textbf{rLin} et \textbf{rReg}.
Nous nous somme particuli\`erement int\'eress\'e aux facteurs li\'es aux param\`etres de l'acquisition, comme la dur\'ees des examens ou encore la fr\'equence d'\'echantillonage, ou aux facteurs d\'ependant de la strat\'egie d'analyse, comme le niveau de bruit dans les cin\'etiques, le nombre de r\'egions d'int\'er\^et ou les caract\'eristiques du tissu de r\'ef\'erence choisi.
Nos \'etudes de simulations ont r\'ev\'el\'e que les param\`etres du mod\`ele \textbf{rLin} sont affect\'es par des biais syst\'ematiques variant d'une r\'egion \`a l'autre.
L'utilisation du mod\`ele \textbf{rReg} permet de r\'eduire les variations r\'egionales du biais d'estimation, en particulier en ce qui concerne les param\`etres $k_T$ et $k_R$, mais aussi les param\`etres $rV$ et $rF$
dans une moindre mesure.
L'estimation de l'approche \textbf{rReg} s'est \'egalement montr\'ee plus robuste et plus homog\`ene que l'estimation de l'approche \textbf{rLin} en cas d'acquistions plus courtes ou moins bien \'echantillonn\'ees.
Enfin, l'impact du tissu de r\'ef\'erence sur la justesse de l'estimation a \'et\'e d\'emontr\'e \`a travers notre \'etude de simulation, mais une \'etude plus approfondie est \`a pr\'evoir afin de mieux en appr\'ehender les m\'echanismes.

\subsection*{Conclusion}
Le mod\`ele \textbf{rReg} s'est av\'er\'e \^etre prometteur pour quantifier la perfusion, et nous avons d\'emontr\'e son applicabilit\'e en \'echographie de contraste dans ce manuscrit.
Lorsque l'on cherche \`a quantifier la perfusion dans plusieurs r\'egions d'int\'er\^et, nous recommandons de prendre en compte les relations entre les param\`etres r\'egionaux pour \'eviter les incoh\'erences entre r\'egions, tout en rendant l'estimation plus robuste.
Le mod\`ele peut-\^etre appliqu\'e \`a une \'echelle plus fine, i.e.~\`a l'\'echelle du pixel ou du macro-pixel, pour mieux r\'ev\'eler l'h\'et\'erog\'en\'eit\'e fonctionnelle du tissu \'etudi\'e.
Cependant l'absence de correspondance pixel \`a pixel dans notre \'etude test-retest nous a pouss\'e \`a utiliser un d\'ecoupage r\'egional, permettant ainsi une comparaison des param\`etres.

La majorit\'e des acquisitions d'\'echographie de contraste se font encore en 2D, rendant ainsi la comparaison d'examens difficile.
Il est en effet difficile de s'assurer que le m\^eme plan soit imag\'e dans deux examens acquis \`a quelques jours d'intervalle, et c'est encore plus difficile lorsque l'on \'etudie l'\'evolution de tissu dont la forme et la taille changent entre les examens, e.g.~croissance tumorale, r\'eponse \`a un traitement.
Nous recommandons donc l'usage de donn\'ees 3D lorsque cela est possible, en particulier lorsque l'on cherche \`a comparer des examens.
Les d\'eveloppements r\'ecents des \'echographes 3D est donc prometteur pour les applications de suivi tumoral.
En effet l'\'echographie sera alors capable de rattraper les modalit\'es d'imagerie tomographiques, donnant ainsi acc\`es \`a des informations plus pertinentes sur la forme, la taille, la structure et la fonction des l\'esions, tout en permettant une imagerie en temps-r\'eel, non-ionisante et peu couteuse.

L'imagerie de contraste 3D est courante en tomographie par \'emission de positrons (TEP), en scanner \`a rayons X (CT), ou en imagerie par r\'esonnance magn\'etique (IRM).
La capacit\'e du mod\`ele \textbf{rReg} \`a \^etre appliqu\'e \`a ces modalit\'es d'imagerie doit-\^etre \'etudi\'ee plus en d\'etails, et n\'ecessitera parfois des adaptations pour prendre en compte les caract\'eristiques des donn\'ees, du tissu \'etudi\'e ou encore de l'agent de contraste.
En effet, l'adaptation de la m\'ethode d'estimation r\'egularis\'ee \`a diverses architectures de mod\`eles compartimentaux doit-\^etre \'etudi\'ee afin de prendre en compte les caract\'eristiques du tissu et de l'agent de contraste.
Par ailleurs, l'\'etude de la perfusion dans le foie n\'ecessiterait d'autres adaptations du mod\`ele, notamment la prise en compte d'une entr\'ee portale en plus de l'entr\'ee art\'erielle.
Les structures vasculaires dans le rein \'etant complexes et se traduisant par plusieurs phases de perfusion, le mod\`ele pourrait-\^etre adapt\'e pour les prendre en compte.
En effet l'usage du rein comme tissu de r\'ef\'erence, en prenant en compte les diff\'erentes phases, pourrait encore am\'eliorer la qualit\'e de l'estimation.

L'impact du tissu de r\'ef\'erence sur les param\`etres de perfusion du mod\`ele \textbf{rReg} a \'et\'e d\'emontr\'ee dans le Chapitre~\ref{chapter:PLOSONE}, cependant une \'etude approfondie est n\'ecessaire afin de mieux d\'efinir les caract\'eristiques du tissu de r\'ef\'erence id\'eal, mais aussi pour mieux comprendre l'impact d'un tissu de r\'ef\'erence non-id\'eal.
Inclure plusieurs tissus de r\'ef\'erence dans le mod\`ele pourrait rendre l'estimation des param\'etres plus robustes aux caract\'eristiques des tissus de r\'ef\'erence.

Pour conclure, le choix d'une m\'ethode de quantification de la perfusion reste une t\^ache difficile qui d\'epend fondamentalement des donn\'ees et du but de l'\'etude.
La comparaison d'examens est particuli\`erement difficile en raison des variations exp\'erimentales et physiologiques qui se produisent entre les examens.
Cette th\`ese a d\'emontr\'e la capacit\'e du mod\`ele relatif \`a un compartiment \`a quantifier la perfusion \`a partir de donn\'ees d'\'echographie de contraste de fa\c{c}on reproductible et robuste \`a l'\'echelle r\'egionale, r\'ev\'elant ainsi l'h\'et\'erog\'en\'eit\'e fonctionnelle de tumeurs et rendant possible les comparaisons intra-examen et inter-examen.
Nous insistons sur la n\'ecessit\'e de prendre en compte les relations entre les param\`etres de perfusion des diff\'erents tissus \'etudi\'es, puisque nous avons d\'emontr\'e l'impact de cette consid\'eration sur la justesse, la robustesse et la reproductibilit\'e de l'estimation des param\`etres dans le mod\`ele \textbf{rReg}.
N\'eanmoins, l'usage de tissus de r\'ef\'erence pour suivre l'\'evolution de la perfusion de tumeurs recevant un traitement soul\`eve une question quant \`a l'effet des traitement sur le tissus de r\'ef\'erence, notamment en ce qui concerne les traitements anti-angiog\'eniques.
Les mod\`es utilisant un tissu de r\'ef\'erence se sont montr\'e plus reproductibles et robustes que les mod\`eles utilisant une fonction d'entr\'ee art\'erielle, mais qu'en sera-t-il une fois les difficult\'es rencontr\'ees dans l'estimation de la fonction d'entr\'ee art\'erielle surmont\'ees ?

\end{otherlanguage}