\begin{otherlanguage}{french}
\subsection*{Titre: Quantification de la perfusion tissulaire en \'echographie de contraste: vers la comparaison robuste d'examens}

\subsection*{Introduction}
Cette th\`ese effectu\'e au sein du {\em Laboratoire d'Imagerie Biom\'edicale} (LIB) a \'et\'e financ\'e par la {\em Fondation pour la Recherche M\'edicale} (FRM).
Le projet global consiste \`a d\'evelopper un outil de classification multi-param\'etrique des tissus tumoraux exploitant diverses modalit\'es d'imagerie ultrasonore, i.e.~\'echographie quantitative, elastosonographie et \'echographie de contraste.
Les donn\'es serviront au d\'eveloppement d'un mod\`ele r\'ealiste de croissance tumorale ainsi que de la r\'eponse aux traitements anti-tumoraux.
La premi\`ere \'etape de ce projet consistait donc \`a estimer de fa\c{c}on pr\'ecis\'ement et reproductible des param\`etres de perfusion \`a partir de donn\'es de contrast ultrasonores, ce afin de les utiliser dans un contexte de suivi longitudinal.

La quantification de la perfusion est une t\^ache difficile, en effet des variations peuvent intervenir entre les examens, que ce soit au niveau exp\'erimental ou physiologique. 
Cependant ce processus s'av\`ere crucial lorsque l'on cherche \`a \'etudier la croissance de tumeurs, avec ou sans traitement.
L'imagerie de contraste permet d'\'etudier la perfusion in-vivo, cependant la comparaison quantitative d'examens reste difficile en raison du manque de reproductibilit\'e des acquisitions.
La majorit\'e des m\'ethodes de quantification de la perfusion ont \'et\'e d\'evelopp\'ees pour analyser des donn\'ees \`a une \'echelle globale, ce qui soit masque les variations spatiale de la perfusion tissulaire, soit n\'eglige les relations entre les param\`etres locaux.
Le but de cette th\`ese est de rendre l'estimation de param\`etres de perfusion robustes aux variations inter-examens, afin de rendre possible la comparaison d'examens tout en r\'ev\'elant l'h\'et\'erog\'en\'eit\'e spatiale de la fonction vasculaire.
Notre \'etude se concentre sur la quantification de l'\'echographie de contraste, cependant l'usage des m\'ethodes propos\'ees pour la quantification d'autres modalit\'es d'imagerie de contraste pourrait-\^etre \'etudi\'ee.

Le manuscrit est divis\'e en trois parties.
La premi\`ere partie cherche \`a \'etablir l'\'etat de l'art des m\'ethodes de quantification de la perfusion d\'evelopp\'ees pour les diff\'erentes modalit\'es d'imagerie de contraste.
La seconde partie \'etudie la reproductibilit\'e des param\`etres obtenus \`a l'aide de diff\'erentes approches ainsi que les relations qui les unissent, i.e.~une approche semi-quantitative, un mod\`ele \`a un compartiment utilisant une fonction d'entr\'e art\'erielle, et un mod\`le \`a un compartiment utilisant un tissus de r\'ef\'erence.
Enfin, dans la troisi\`eme partie nous pr\'esentons une nouvelle approche d'estimation du mod\`ele \`a un compartiment utilisant un tissus de r\'ef\'erence exploitant et r\'ev\'elant l'h\'et\'erog\'en\'eit\'e spatiale de la fonction vasculaire.

\subsection*{Partie I.~Quantification de la perfusion: \'etat de l'art}
Dans cette premi\`ere partie, nous \'etablissons un \'etat de l'art des m\'ethodes d\'evelopp\'ees pour quantifier la perfusion tumorale. 
Les m\'ethodes de quantification sont class\'ees en trois cat\'egories : semi-quantitatives, d\'econvolution, ou compartimentales.
Les approches semi-quantitatives extraient des param\'etres caract\'erisant la cin\'etique de la concentration en agent de contraste et sont courrament utilis\'es pour caract\'eriser la perfusion tissulaire, mais les param\`tres de ces mod\`les n'ont pas de lien direct avec la physiologie.
Les approches de d\'econvolution, ainsi que la majorit\'e des approches compartimentales, n\'ecessitent la connaissance de la fonction d'entr\'ee art\'erielle.
La fonction d'entr\'ee art\'erielle peut-\^etre obtenue par pr\'el\`evement sanguin, ou directement dans l'image. 
Cependant, les pr\'el\`evements sanguins sont invasifs, en particulier les pr\'el\`evements art\'eriels, et en raison des fortes concentrations en agent de contraste observ\'ees dans les art\`eres l'estimation directe de la fonction d'entr\'e art\'erielle dans l'image est affect\'ee par des artefacts, y compris des effets de saturation et de volumes partiels.
Les difficult\'es rencontr\'ees lors de l'estimation de la fonction d'entr\'ee art\'erielle ont motiv\'e le d\'eveloppement de m\'ethodes utilisant un tissus de r\'ef\'erence, permettant la quantification relative de la perfusion.
En effet, un tissus de r\'ef\'erence peut \^etre choisi dans une r\'egion \'etendue et bien perfus\'e de l'image, r\'eduisant ainsi les risques de saturation et de volumes partiels.
\`A notre connaissance, es mod\`eles compartimentaux, qu'ils utilisent une fonction d'entr\'e art\'erielle ou un tissus de r\'ef\'erence, ont \'et\'e appliqu\'es \`a la quantifiction de la perfusion dans la plupart des modalit\'es d'imagerie de contraste, \`a l'exception de l'\'echographie.

\subsection*{Partie II.~Reproductibilit\'e des m\'ethodes existantes et les relations entre les diff\'erentes approches}

\subsection*{Partie III.~Proposition et \'evaluation d\'une nouvelle m\'ethode de quantification}

\subsection*{Conclusion}


\end{otherlanguage}