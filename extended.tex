\begin{otherlanguage}{francais}
\subsection*{Introduction}
Cette th\`ese effectu\'ee au sein du {\em Laboratoire d'Imagerie Biom\'edicale} (LIB) a \'et\'e financ\'ee par la {\em Fondation pour la Recherche M\'edicale} (FRM).
Le projet global consiste \`a d\'evelopper un outil de classification multi-param\'etrique des tissus tumoraux exploitant diverses modalit\'es d'imagerie ultrasonore, i.e.~l'\'echographie quantitative, l'\'elastosonographie et l'\'echographie de contraste.
Les donn\'ees serviront au d\'eveloppement d'un mod\`ele r\'ealiste de croissance tumorale ainsi que de la r\'eponse aux traitements anti-tumoraux.
Une premi\`ere \'etape cl\'e de ce projet sur laquelle portait mon travail de th\`ese a consist\'e \`a estimer de fa\c{c}on pr\'ecise et reproductible des param\`etres de perfusion \`a partir de donn\'ees de contraste ultrasonore, ce afin de les utiliser dans un contexte de suivi longitudinal et pour mod\'eliser l'\'evolution tumorale.

La quantification de la perfusion est une t\^ache difficile, en effet des variations peuvent intervenir entre les examens, que ce soit au niveau exp\'erimental ou physiologique. 
Ce processus s'av\`ere n\'eanmoins crucial pour \'etudier la croissance de tumeurs, avec ou sans traitement.
Si l'imagerie de contraste permet d'\'etudier la perfusion in-vivo, la comparaison quantitative d'examens reste difficile en raison du manque de reproductibilit\'e des acquisitions.
De nombreuses m\'ethodes de quantification de la perfusion ont \'et\'e d\'evelopp\'ees pour analyser des donn\'ees \`a une \'echelle globale, ce qui masque les variations spatiales de la perfusion tissulaire, et ne permet pas d'exploiter les relations entre les param\`etres locaux.
Le but de cette th\`ese est de rendre l'estimation de param\`etres de perfusion robustes aux variations inter-examens, afin de rendre possible la comparaison d'examens tout en r\'ev\'elant l'h\'et\'erog\'en\'eit\'e spatiale de la perfusion.
Notre \'etude se concentre sur la quantification en \'echographie de contraste, cependant les m\'ethodes propos\'ees pourrait-\^etre \'etudi\'ee pour la quantification avec d'autres modalit\'es d'imagerie de contraste.

Le manuscrit est divis\'e en trois parties.
La premi\`ere partie cherche \`a \'etablir un \'etat de l'art des m\'ethodes de quantification de la perfusion d\'evelopp\'ees pour les diff\'erentes modalit\'es d'imagerie de contraste.
La seconde partie \'etudie la reproductibilit\'e des param\`etres obtenus \`a l'aide de diff\'erentes approches ainsi que les relations qui unissent les diff\'erents param\`etres estim\'e avec ces approches.
En particulier sont compar\'es une approche semi-quantitative, un mod\`ele \`a un compartiment aliment\'e par une fonction d'entr\'e art\'erielle, et un mod\`ele \`a un compartiment utilisant un tissu de r\'ef\'erence.
Enfin, dans la troisi\`eme partie nous pr\'esentons une nouvelle approche d'estimation du mod\`ele \`a un compartiment utilisant un tissu de r\'ef\'erence exploitant l'h\'et\'erog\'en\'eit\'e spatiale des motifs de perfusion au sein de la tumeur.

\subsection*{Partie I.~Quantification de la perfusion: \'etat de l'art}
Dans cette premi\`ere partie, compos\'ee du Chapitre~\ref{chapter:review} nous \'etablissons un \'etat de l'art des m\'ethodes d\'evelopp\'ees pour quantifier la perfusion tumorale.
Ce Chapitre pr\'esente une s\'election de m\'ethodes de quantification fondatrices, ainsi que leur \'evolution \`a travers les diff\'erentes modifications qui y ont \'et\'e apport\'ees pour surmonter diff\'erentes limites.
Leur usage est d\'eclin\'e pour chaque modalit\'e d'imagerie (TEP, TDM, IRM et \'echographie).

Les m\'ethodes de quantification ont \'et\'e class\'ees en trois cat\'egories : semi-quantitatives, d\'econvolution, ou compartimentales.
Les approches semi-quantitatives extraient des param\'etres caract\'erisant la cin\'etique de la concentration en agent de contraste et sont courrament utilis\'es pour caract\'eriser la perfusion tissulaire, mais les param\`etres de ces mod\`eles n'ont pas de lien direct avec la physiologie.
Les approches de d\'econvolution, ainsi que la majorit\'e des approches compartimentales, n\'ecessitent la connaissance de la fonction d'entr\'ee art\'erielle.
La fonction d'entr\'ee art\'erielle peut-\^etre obtenue par pr\'el\`evements sanguins, ou directement dans l'image. 
Cependant, les pr\'el\`evements sanguins sont invasifs, en particulier les pr\'el\`evements art\'eriels, et en raison des fortes concentrations en agent de contraste observ\'ees dans les art\`eres l'estimation directe de la fonction d'entr\'ee art\'erielle dans l'image est affect\'ee par des artefacts, notamment les effets de saturation et de volume partiel.

Les difficult\'es rencontr\'ees lors de l'estimation de la fonction d'entr\'ee art\'erielle ont motiv\'e le d\'eveloppement de m\'ethodes utilisant un tissu de r\'ef\'erence, permettant la quantification relative de la perfusion.
En effet, un tissu de r\'ef\'erence peut \^etre choisi dans une r\'egion plus grande qu'une art\`ere et bien perfus\'ee, r\'eduisant ainsi les risques de saturation et de volume partiel.
Des mod\`eles compartimentaux sont courrament utilis\'es pour quantifier la perfusion et d'autres m\'ecanismes plus complexes incluant diffusion extravasculaire et m\'etabolisme en TEP, en TDM et en IRM.
Cette partie s'ach\`eve sur le constat que si un certain nombre de transferts ont \'et\'e op\'er\'es d'une modalit\'e d'imagerie \`a une autre, \`a notre connaissance les mod\`eles compartimentaux, qu'ils utilisent une fonction d'entr\'ee art\'erielle ou un tissu de r\'ef\'erence, n'ont pas \'et\'e appliqu\'es \`a la quantification de la perfusion en \'echographie de contraste.

\subsection*{Partie II.~Reproductibilit\'e des m\'ethodes de quantification existantes en \'echographie de contraste et les relations entre les diff\'erentes m\'ethodes}
Cette partie est compos\'ee du Chapitre~\ref{chapter:PMB} dans lequel nous avons \'etudi\'e la reproductibilit\'e des param\`etres de perfusion estim\'es par diff\'erentes m\'ethodes de quantification sur des donn\'ees d'\'echographie de contraste, et du Chapitre~\ref{chapter:PMB2} qui compl\`ete le Chapitre pr\'ec\'edent en \'etablissant les relations th\'eoriques et empiriques entre les param\`etres des diff\'erents mod\`eles, qui permettend de justifier pleinement le choix des param\`etres s\'electionn\'es dans l'\'etude de reproductibilit\'e.

L'\'etude pr\'esent\'ee dans le Chapitre~\ref{chapter:PMB} a fait l'objet d'une publication dans {\em Physics and Medicine in Biology} sous le titre \``Quantification of tumor perfusion using dynamic contrast-enhanced ultrasound: impact of mathematical modeling\''.
Nous y pr\'esentons une \'etude de reproductibilit\'e des param\`etres de perfusion en \'echographie de contraste r\'ealis\'ee sur quatre souris porteuses de tumeurs colorectales.
Chaque souris a subi quatre examens test-retest d'\'echographie de contraste avec injection automatique d'un bolus de microbulles suivant un protocole d\'evelopp\'e au laboratoire.
Les acquisitions \'etaient espac\'ees de quinze minutes afin de garantir l'\'elimination totale des microbulles avant chaque nouvelle acquisition.
La r\'egion perfus\'ee des tumeurs (en excluant la partie n\'ecrotique) a ensuite \'et\'e segment\'ee en 32 r\'egions, permettant la d\'efinition de 32 cin\'etiques.
Ce d\'ecoupage r\'egional a \'et\'e choisi afin de montrer l'h\'et\'erog\'en\'eit\'e de la perfusion tumorale tout en garantissant un rapport signal sur bruit suffisant permettant la quantification.
Enfin, diverses m\'ethodes de quantification ont \'et\'e appliqu\'ees \`a ces cin\'etiques r\'egionales afin d'estimer des param\`etres de perfusion et il a \'et\'e possible de les comparer.

En particulier, nous avons compar\'e la reproductibilit\'e des param\`etres de perfusion estim\'es par les m\'ethodes suivantes: (a)~une approche bas\'ee sur le mod\`ele log-normal (\textbf{LN}) estimant des param\`etres semi-quantitatifs, notamment l'aire sous la courbe $AUC$ et le taux de remplissage $WIR$, (b)~un mod\`ele \`a un compartiment utilisant une fonction d'entr\'ee art\'erielle (\textbf{AIF}) estimant des param\`tres absolus, i.e.~le volume sanguin tissulaire $V$, le flux sanguin tissulaire $F$, et par d\'eduction la constante de transfert $k_T = \frac{F}{V}$, (c)~un mod\`ele \`a un compartiment utilisant un tissu de r\'ef\'erence (\textbf{RT}), conduisant \`a l'estimation de param\`etres relatifs, le volume sanguin tissulaire relatif $rV_{RT}$, le flux sanguin tissulaire relatif $rF_{RT}$, ainsi que la constante de transfert dans le tissu d'int\'er\^et $k_{T}$, et par d\'eduction la constante de transfert dans le tissu de r\'ef\'erence $k_R = \frac{rV \cdot k_T}{rF}$.
Dans cette \'etude, les trois mod\`eles ont \'et\'e ajust\'e aux cin\'etiques de perfusion exp\'erimentales en minimisant l'erreur au sens des moindres carr\'es \`a l'aide d'algorithmes de minimisation non-lin\'eaires.
La valeur du param\`etre $k_R$ a \'et\'e fix\'ee de fa\c{c}on syst\'ematique \`a une valeur estim\'ee \`a partir d'une mod\'elisation compartimentale avec fonction d'entr\'ee art\'erielle.
En effet ce param\`etre n'est pas identifiable dans la formulation non-lin\'eaire du mod\`ele.

Afin d'\'etudier l'effet de la normalisation sur la reproductibilit\'e des param\`etres de perfusion, nous avons d\'efini des param\`etres relatifs \`a partir des param\`etres du mod\`ele \textbf{LN}, not\'es $rAUC$ et $rWIR$, et des param\`etres absolus du mod\`ele monocompartimental avec fonction d'entr\'ee art\'erielle \textbf{AIF}, not\'es $rV_{AIF}$ et $rF_{AIF}$, en les normalisant par leur valeur dans le tissu de r\'ef\'erence pr\'ec\'edemment d\'ecrit.
Enfin, la prise en compte du temps d'arriv\'ee du bolus dans le tissu \'etudi\'e \'etant souvent n\'eglig\'ee dans les autres modalit\'es d'imagerie, nous avons \'etudi\'e son impact sur la qualit\'e de mod\'elisation des donn\'ees de perfusion.

La n\'ecessit\'e de prendre en compte le temps d'arriv\'ee du bolus dans le tissu d'int\'er\^et a \'et\'e d\'emontr\'ee en \'etudiant la qualit\'e de la mod\'elisation obtenue avec les diff\'erents mod\`eles compartimentaux avec et sans estimation d'un temps de retard. 
En effet, lorsqu'aucun temps de retard n'est pris en compte dans les mod\`eles, le nombre de r\'egions pr\'esentant une mod\'elisation de mauvaise qualit\'e est \'egal \`a 212 pour le mod\`ele \textbf{AIF} et \`a 56 pour le mod\`ele \textbf{RT}, sur un total de 512 r\'egions \'etudi\'ees dans l'\'etude test-retest ($4 \times 4 \times 32$).
Ce nombre s'\'el\`eve \`a 19 quel que soit le mod\`ele (\textbf{AIF} ou \textbf{RT}) lorsque le temps de retard entre l'art\`ere et la r\'egion d'int\'er\^et est pris en compte.

Sur les \'etudes test-retest, les param\`etres absolus du mod\`ele \textbf{LN} se sont montr\'e peu reproductibles avec des coefficients de variation m\'edians pour les 32 r\'egions de l'ordre de 30\% pour $AUC$ et 40\% pour $WIR$, sugg\'erant une forte sensibilit\'e des param\`etres aux variations inter-examens.
Les param\`etres absolus du mod\`ele \textbf{AIF} se sont r\'ev\'el\'es plus reproductibles avec des coefficients de variation m\'edians de l'ordre de 25\% pour $V$ et 35\% pour $F$.
Cependant l'\'etude a d\'emontr\'e la sensibilit\'e de ces param\`etres \`a la fonction d'entr\'ee art\'erielle utilis\'ee, et les difficult\'es rencontr\'ees lors de son estimation dans l'image en raison de la petite taille de l'art\`ere dans le champ de vue et des fortes concentrations en agent de contraste observ\'ees dans les gros vaisseaux.
La normalisation des param\`etres des mod\`eles \textbf{LN} et \textbf{AIF}, donnant ainsi naissance aux mod\`eles \textbf{rLN} et \textbf{rAIF}, a permis une r\'eduction significative de la variabilit\'e inter-examen.
En effet, les param\`etres li\'es au volume sanguin tissulaire, $rAUC$ et $rV_{AIF}$, ont des coefficients de variation m\'edians de l'ordre de 20\%, tandis que les param\`etres li\'es au flux sanguin tissulaire, $rWIR$ et $rF_{AIF}$, ont des coefficients de variation m\'edians l\'eg\`erement sup\'erieurs \`a 30\%.
L'estimation directe de param\`etres relatifs \`a l'aide du mod\`ele \textbf{RT} s'est r\'ev\'el\'ee la plus reproductible de l'\'etude, avec des coefficients de variation m\'edians l\'eg\`erement inf\'erieurs \`a 20\% pour $rV_{RT}$ et \`a 30\% pour $rF_{RT}$.

L'impact de la fonction d'entr\'ee sur les valeurs des param\`etres absolus et relatifs du mod\`ele \textbf{AIF} a \'et\'e \'etudi\'e en comparant les valeurs obtenues pour un examen donn\'e, avec deux fonctions d'entr\'ee estim\'ees dans les images suite \`a deux types de seuils.
Les param\`etres absolus se sont montr\'e extr\^emement sensibles \`a la fonction d'entr\'ee art\'erielle, $V$ et $F$ pouvant varier du simple au double dans certaines r\'egions.
Les param\`etres relatifs du mod\`ele \textbf{rAIF} se sont montr\'e plus robustes aux variations de la fonction d'entr\'ee art\'erielle, en particulier $rV_{AIF}$ qui est pratiquement insensible \`a la fonction d'entr\'ee utilis\'ee dans notre \'etude.

Le Chapitre~\ref{chapter:PMB2} pr\'esente les relations existant entre les param\`etres des diff\'erents mod\`eles \'etudi\'es dans le Chapitre~\ref{chapter:PMB}, i.e.~\textbf{LN}, \textbf{rLN}, \textbf{AIF} et \textbf{RT}, sous l'hypoth\`ese d'un mod\`ele compartimental aliment\'e par une fonction d'entr\'ee art\'erielle.
Les relations sont d'abord \'etablies th\'eoriquement en \'etablissant les d\'efinitions analytiques des param\`etres d'un mod\`ele en fonction des param\`etres d'un autre mod\`ele, puis empiriquement \`a travers une \'etude de corr\'elation des param\`etres estim\'es par les diff\'erents mod\`eles.
En particulier, les param\`etres des mod\`eles \textbf{LN} et \textbf{rLN} ont \'et\'e exprim\'es en fonction des param\`etres du mod\`ele \textbf{AIF} pour trois fonctions d'entr\'ee diff\'erentes, i.e.~dans les deux cas id\'ealis\'es que sont l'entr\'ee de type Dirac et de type porte, mais \'egalement dans le cas g\'en\'eral.
Ce chapitre d\'emontre notamment les relations entre le param\`etre $AUC$ et le param\`etre de volume sanguin tissulaire $V$ d'une part, et entre le param\`etre $WIR$ et le param\`etre de flux sanguin tissulaire $F$ d'autre part. 
Ce constat a motiv\'e notre s\'election de param\`etres issus du mod\`ele \textbf{LN} dans le Chapitre~\ref{chapter:PMB} puisque d'autres param\`etres auraient pu \^etre \'etudi\'es (WOR, MTT), mais les relations avec les param\`etres de flux sanguin tissulaire et de volume sanguin tissulaire \'etaient moins directes. 
Mais il montre \'egalement la sensibilit\'e des param\`etres semi-quantitatifs \`a la fonction d'entr\'ee art\'erielle, et confirme l'int\'er\^et de la normalisation par une r\'egion de r\'ef\'erence pour s'absoudre de cette d\'ependance.
L'int\'er\^et de la normalisation est \'egalement confirm\'e empiriquement, en effet la corr\'elation entre les param\`etres relatifs des diff\'erents mod\`eles est beaucoup plus forte que celle entre les param\`etres absolus.

\subsection*{Partie III.~Proposition et \'evaluation d'une nouvelle m\'ethode de quantification}
Dans cette troisi\`eme partie, nous pr\'esentons d'abord une approche d'estimation reposant sur une formulation lin\'eaire du mod\`ele \textbf{RT} qui a \'et\'e propos\'ee initialement pour d'autres modalit\'es d'imagerie de contraste (TEP et IRM), not\'ee \textbf{rLin}.
Les m\'ethodes de r\'esolution lin\'eaires ont l'avantage de proc\'eder \`a l'estimation directe des param\`etres minimisant la fonction d'erreur, permettant ainsi d'\'eviter les probl\`emes li\'es \`a l'initialisation des param\`etres tout en acc\'el\'erant l'estimation.
Par ailleurs, ce mod\`ele permet l'estimation d'un param\`etre suppl\'ementaire, i.e.~la constante de transfert du tissu de r\'ef\'erence $k_R$.
Cependant, cette formulation conduit potentiellement \`a diff\'erentes valeurs de ce param\`etre qui bien \'evidemment devrait \^etre unique.
Pour pallier \`a cet inconv\'enient, nous avons propos\'e une nouvelle m\'ethode d'estimation r\'egularis\'ee, not\'ee \textbf{rReg}, exploitant le fait que plusieurs r\'egions soient analys\'ees conjointement en s'assurant qu'une valeur unique du param\`etre $k_R$ soit estim\'ee pour toutes les r\'egions d'un m\^eme examen.
Nous avons par la suite \'etudi\'e la robustesse de ces deux mod\`eles \textbf{rLin} et \textbf{rReg} aux variations inter-examens, \`a des ph\'enom\`enes physiologiques comme la recirculation, \`a des param\`etres d'acquisition, et \`a diff\'erentes strat\'egies dans l'analyse des donn\'ees. 
\`A notre connaissance ce type d'approche n'a \'et\'e propos\'e dans aucune des modalit\'es d'imagerie de perfusion.

Les \'equations des mod\`eles \textbf{rLin} et \textbf{rReg} sont d'abord d\'evelopp\'ees bri\`evement dans le Chapitre~\ref{chapter:IUS}.
En effet, le Chapitre~\ref{chapter:IUS} qui \'etudie la reproductibilit\'e des param\`etres de perfusion estim\'es avec les deux approches a fait l'objet d'une publication courte suite \`a la pr\'esentation de ces travaux lors du congr\`es {\em IEEE International Ultrasonics Symposium} (IUS) qui s'est tenu \`a Tours en septembre 2016.
Le Chapitre~\ref{chapter:IRBM} \'etudie sp\'ecifiquement l'impact de la recirculation des microbulles sur la justesse et la pr\'ecision des estimations r\'ealis\'ees avec diff\'erents mod\`eles, en utilisant des cin\'etiques de perfusion simul\'ees de fa\c{c}on r\'ealiste  \`a l'aide du mod\`ele \textbf{AIF}.
Cette \'etude a fait l'objet d'une publication intitul\'ee \``Impact of Recirculation in Dynamic Contrast-Enhanced Ultrasound:~A Simulation Study\'' et publi\'ee dans la revue {\em Innovation and Research in BioMedical Engineering} (IRBM).
Dans le Chapitre~\ref{chapter:PLOSONE} sont d\'etaill\'es les d\'eveloppements th\'eoriques qui ont amen\'e le d\'eveloppement du mod\`ele \textbf{rReg}, puis nous avons \'etudi\'e l'impact de divers param\`etres sur la justesse et la pr\'ecision des estimations.
Nous nous somme notamment int\'eress\'es aux param\`etres d'acquisition comme la dur\'ee des examens ou la fr\'equence d'\'echantillonage, mais aussi \`a des strat\'egies d'analyse comme le nombre de r\'egions d'int\'er\^et ou le choix du tissu de r\'ef\'erence.
Nous pr\'evoyons de soumettre une version revue de ce Chapitre pour publication dans la revue {\em Medical Image Analysis}.

Dans le Chapitre~\ref{chapter:IUS}, les param\`etres de perfusion estim\'es \`a l'aide des m\'ethodes \textbf{rLin} et \textbf{rReg} sont compar\'es, en terme de reproductibilit\'e, aux param\`etres des mod\`eles \textbf{LN} et \textbf{rLN}, en se basant la m\^eme \'etude test-retest que celle d\'etaill\'ee dans le Chapitre~\ref{chapter:PMB}.
Le mod\`ele \textbf{rLin} s'est r\'ev\'el\'e \^etre l'approche relative la moins robuste aux variations inter-examens, notamment en terme de param\`etre de flux puisque nous avons obtenu un coefficient de variation m\'edian sup\'erieur \`a 40\% pour $rF_{rLin}$.
Cette forte variabilit\'e montre la faible identifiabilit\'e du param\`etre $k_R$ et son impact sur les autres param\`etres du mod\`ele, notamment sur $rF_{rLin}$.
Elle explique \'egalement la pratique courante qui consiste \`a ne pas estimer $k_R$ mais \`a utiliser une valeur fixe provenant de la litt\'erature ou d'exp\'eriences pr\'ealables pour stabiliser l'estimation des autres param\`etres, comme nous l'avons fait avec le mod\`ele \textbf{RT} pr\'esent\'e dans le Chapitre~\ref{chapter:PMB}.
L'approche r\'egularis\'ee \textbf{rReg} s'est montr\'ee la plus reproductible dans cette \'etude, puisque nous avons obtenu des coefficients de variation m\'edians inf\'erieurs \`a 20\% pour $rV_{rReg}$, et inf\'erieurs \`a 30\% pour $rF_{rReg}$, dans notre \'etude test-retest.

Dans le Chapitre~\ref{chapter:IRBM}, nous avons \'etudi\'e la sensibilit\'e \`a la recirculation des param\`etres estim\'es \`a l'aide des mod\`eles \textbf{LN}, \textbf{rLN}, \textbf{rLin} et \textbf{rReg}.
En effet la recirculation de l'agent de contraste dans le tissu \'etudi\'e est un probl\`eme connu mais rarement pris en compte dans les approches de quantification de la perfusion in vivo en \'echographie de contraste.
Pour s'absoudre de ce ph\'enom\`ene, une approche simpliste consiste \`a \``couper\'' les donn\'ees avant que la recirculation n'intervienne.
Nous avons donc \'egalement \'etudi\'e la capacit\'e de cette technique \`a estimer des param\`etres de perfusion en l'appliquant aux mod\`eles \textbf{LN} et \textbf{rLN}, dont les variantes sont not\'ees \textbf{LN$_{WI}$} et \textbf{rLN$_{WI}$}.

Les param\`etres absolus du mod\`ele \textbf{LN} se sont r\'ev\'el\'es particuli\`erement sensibles \`a la recirculation en raison de leur forte d\'ependance \`a la fonction d'entr\'ee art\'erielle.
En ce qui concerne le mod\`ele relatif \textbf{rLN}, l'estimation du param\`etre $rAUC$ s'est montr\'ee robuste \`a la recirculation, cependant le param\`etre $rWIR$ reste sensible \`a la recirculation.
Les approches \textbf{LN$_{WI}$} et \textbf{rLN$_{WI}$} se sont r\'ev\'el\'ees peu robustes, en effet malgr\'e l'accord entre les param\`etres estim\'es avec et sans recirculation, le nombre r\'eduit d'\'echantillons utilis\'es durant l'ajustement du mod\`ele ne permet pas d'ajuster le mod\`ele correctement \`a la phase descendante de la cin\'etique, et rend l'estimation particuli\`erement sensible au bruit.

Les param\`etres de flux et de volume sanguin tissulaires relatifs des mod\`eles \textbf{rLin} et \textbf{rReg} se sont r\'ev\'el\'es peu sensibles \`a la recirculation, avec un l\'eger avantage pour le mod\`ele \textbf{rLin}, ce qui confirme la robustesse structurelle des mod\`eles compartimentaux \`a cet \'egard.
En revanche, l'estimation des param\`etres $k_T$ et $k_R$ par le mod\`ele \textbf{rLin} est sujette \`a de fortes erreurs variant d'une r\'egion \`a l'autre, et ce m\^eme en l'absence de recirculation.
Le mod\`ele \textbf{rReg} n'est pas affect\'e par ce ph\'enom\`ene et montre une erreur d'estimation tr\`es faible pour ces deux param\`etres.
En outre, si les param\`etres $rV_{rReg}$ et $rF_{rReg}$ sont globalement plus affect\'es par la recirculation que les param\`etres $rV_{rLin}$ et $rF_{rLin}$, l'erreur est plus homog\`ene entre les diff\'erentes r\'egions d'int\'er\^et en utilisant le mod\`ele \textbf{rReg}, permettant ainsi une comparaison des param\`etres r\'egionaux plus robuste.

Dans le Chapitre~\ref{chapter:PLOSONE}, nous avons \'etudi\'e l'impact de diff\'erents facteurs, pouvant \^etre la source d'erreurs de quantification, sur les param\`etres de perfusion des mod\`eles \textbf{rLin} et \textbf{rReg}.
Nous nous somme int\'eress\'es aux facteurs li\'es aux param\`etres de l'acquisition, comme le niveau de bruit dans les cin\'etiques, la dur\'ee des examens ou la fr\'equence d'\'echantillonage, ou aux facteurs d\'ependant de la strat\'egie d'analyse, comme le nombre de r\'egions d'int\'er\^et ou les caract\'eristiques du tissu de r\'ef\'erence choisi.
Nos \'etudes de simulation ont montr\'e que les param\`etres du mod\`ele \textbf{rLin} sont affect\'es par des biais syst\'ematiques variant d'une r\'egion \`a l'autre.
L'utilisation du mod\`ele \textbf{rReg} permet de r\'eduire les variations r\'egionales du biais d'estimation, en particulier en ce qui concerne les param\`etres $k_T$ et $k_R$, mais aussi les param\`etres $rV$ et $rF$
dans une moindre mesure.
L'estimation de l'approche \textbf{rReg} s'est \'egalement montr\'ee plus robuste et plus homog\`ene que l'estimation de l'approche \textbf{rLin} en cas d'acquistions plus courtes ou moins bien \'echantillonn\'ees.
Enfin, l'impact du choix du tissu de r\'ef\'erence sur la justesse de l'estimation a \'et\'e d\'emontr\'e \`a travers notre \'etude de simulation, mais une \'etude plus approfondie reste \`a mener afin de mieux en appr\'ehender les m\'ecanismes.

\subsection*{Conclusion}
Le mod\`ele \textbf{rReg} s'est av\'er\'e \^etre prometteur pour quantifier la perfusion, et nous avons d\'emontr\'e son applicabilit\'e en \'echographie de contraste dans ce manuscrit.
Si l'objectif est de quantifier la perfusion dans plusieurs r\'egions d'int\'er\^et (ces r\'egions pouvant \^etre de taille variable), nous recommandons de prendre en compte les relations existant entre les diff\'erents param\`etres r\'egionaux pour \'eviter les incoh\'erences entre r\'egions, tout en rendant l'estimation plus robuste.
Le mod\`ele \textbf{rReg} peut-\^etre appliqu\'e \`a une \'echelle plus fine, i.e.~\`a l'\'echelle du pixel ou du macro-pixel, pour mieux \'etudier l'h\'et\'erog\'en\'eit\'e fonctionnelle du tissu \'etudi\'e.
Cependant l'absence de correspondance pixel \`a pixel dans notre \'etude test-retest nous a pouss\'e \`a utiliser un d\'ecoupage r\'egional, permettant ainsi une comparaison des param\`etres d'une \'etude \`a l'autre.

La majorit\'e des acquisitions d'\'echographie de contraste se font encore en 2D, rendant ainsi la comparaison d'examens difficile.
Il est en effet impossible de s'assurer que le m\^eme plan soit imag\'e dans deux examens acquis \`a quelques jours d'intervalle, et c'est encore plus difficile lorsque l'on \'etudie l'\'evolution d'un tissu (type tissu tumoral) dont la forme et la taille changent entre les examens (li\'e \`a la croissance tumorale ou \`a la r\'eponse \`a une th\'erapie).
Nous recommandons donc l'usage de donn\'ees 3D lorsque cela est possible, en particulier lorsque l'on cherche \`a r\'ealiser un suivi longitudinal.
Les d\'eveloppements r\'ecents en \'echographie 3D sont donc prometteurs pour les applications de suivi tumoral.
En effet l'\'echographie de contraste sera alors capable de rattraper les modalit\'es d'imagerie tomographiques, donnant ainsi acc\`es \`a des informations plus pertinentes sur la forme, la taille, la structure et la fonction des l\'esions, tout en permettant une imagerie en temps-r\'eel, non-ionisante et peu co\^uteuse.

L'imagerie de contraste 3D est syst\'ematique en TEP avec une r\'esolution isotrope, elle est plus limit\'e en TDM, et se d\'eveloppe en IRM.
La capacit\'e du mod\`ele \textbf{rReg} \`a \^etre appliqu\'e \`a ces modalit\'es d'imagerie doit-\^etre \'etudi\'ee plus en d\'etails, et n\'ecessitera parfois des adaptations pour prendre en compte les caract\'eristiques du tissu \'etudi\'e et de l'agent de contraste ou du traceur inject\'e.
En effet, l'adaptation de la m\'ethode d'estimation r\'egularis\'ee \`a diverses architectures de mod\`eles compartimentaux doit-\^etre r\'ealis\'ee.
Par ailleurs, l'\'etude de la perfusion dans le foie n\'ecessiterait d'autres adaptations du mod\`ele, notamment par la prise en compte d'une entr\'ee portale en plus de l'entr\'ee art\'erielle.
Les structures vasculaires dans le rein se traduisant par plusieurs \``phases\'' de perfusion dues \`a la superposition au niveau macroscopique de diff\'erentes structures vascularis\'ees de fa\c{c}on tr\`es diff\'erentes, le mod\`ele doit \^etre adapt\'e pour les prendre en compte.
En effet l'utilisation du rein comme tissu de r\'ef\'erence, en prenant en compte les diff\'erentes phases, pourrait encore am\'eliorer la qualit\'e de l'estimation.

L'impact du tissu de r\'ef\'erence sur les param\`etres de perfusion du mod\`ele \textbf{rReg} a \'et\'e d\'emontr\'e dans le Chapitre~\ref{chapter:PLOSONE}, cependant une \'etude approfondie est n\'ecessaire pour mieux d\'efinir les caract\'eristiques du tissu de r\'ef\'erence id\'eal, mais aussi pour comprendre l'impact du choix d'un tissu de r\'ef\'erence non-id\'eal.
Inclure plusieurs tissus de r\'ef\'erence dans le mod\`ele pourrait rendre l'estimation des param\'etres plus robustes aux caract\'eristiques des tissus de r\'ef\'erence.

Pour conclure, le choix d'une m\'ethode de quantification de la perfusion reste une t\^ache difficile qui d\'epend des donn\'ees et du but de l'\'etude.
La comparaison d'examens est particuli\`erement difficile en raison des variations exp\'erimentales et physiologiques qui se produisent entre les examens.
Cette th\`ese a d\'emontr\'e la capacit\'e du mod\`ele relatif \`a un compartiment \`a quantifier la perfusion \`a partir de donn\'ees d'\'echographie de contraste de fa\c{c}on reproductible et robuste \`a l'\'echelle r\'egionale, r\'ev\'elant ainsi l'h\'et\'erog\'en\'eit\'e fonctionnelle des tumeurs \'etudi\'ees et rendant plus robustes les comparaisons intra-examen et inter-examen.
Il est absolument n\'ecessaire de prendre en compte les relations entre les param\`etres de perfusion des diff\'erents tissus \'etudi\'es, puisque nous avons d\'emontr\'e l'int\'er\^et de cette approche sur la justesse, la robustesse et la reproductibilit\'e de l'estimation des param\`etres dans le mod\`ele \textbf{rReg}.
Il faut noter que les mod\`eles utilisant un tissu de r\'ef\'erence se sont montr\'e plus reproductibles et robustes que les mod\`eles utilisant une fonction d'entr\'ee art\'erielle sur les donn\'ees que sur lesquelles nous avons travaill\'e, mais ce point sera peut-\^etre remis en cause si les difficult\'es rencontr\'ees dans l'estimation de la fonction d'entr\'ee art\'erielle arrivent \`a \^etre surmont\'ees.
Enfin notons que sous th\'erapie, l'usage de tissus de r\'ef\'erence pour suivre l'\'evolution de la perfusion de tumeurs sous th\'erapie soul\`eve une question quant \`a l'effet des traitements sur le tissus de r\'ef\'erence, notamment en ce qui concerne les traitements anti-angiog\'eniques.

Dans la continuit\'e de ces travaux, nous souhaitons confronter les param\`etres de perfusion estim\'es par le mod\`ele \textbf{rReg} aux r\'esultats obtenus en histologie pour valider la m\'ethode.
Les param\`etres de perfusion du mod\`ele \textbf{rReg} pourraient \^etre utilis\'es pour r\'ealiser une classification des tissus tumoraux en trois classes, selon qu'ils soient n\'ecrotiques, hypoxiques ou prolif\'erants.
Les r\'esultats de cette classification alimenteront alors un mod\`ele r\'ealiste de croissance tumrale prenant en compte la r\'eponse th\'erapeutique.

\end{otherlanguage}