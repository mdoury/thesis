\usepackage{url}
\usepackage{textcomp}
\usepackage{verbatim}
\usepackage{amsmath}
% \usepackage{amsfonts}
% \usepackage{amssymb}    % if you want extra symbols
\usepackage{nicefrac}
\usepackage{mathrsfs}
\usepackage{program}
\usepackage{newlfont}
\usepackage{rotating}
\usepackage{varioref}
\usepackage{graphicx}
\usepackage{subfloat}
%\usepackage{txfont}
\usepackage{makeidx}
%\usepackage{tocbibind}
\usepackage{program}
\usepackage{import}
\usepackage{subfigure}
\usepackage{verbatim}
\usepackage{colortbl}
\usepackage{morefloats}
\usepackage[francais,english]{babel}

% \usepackage[LY1]{fontenc}
% \usepackage{patchcmd}
% \usepackage{myfss}
% \usepackage{caslon}
\usepackage{MnSymbol}
\usepackage[mathlf,textlf,minionint]{MinionPro}
\usepackage[T1]{fontenc}
\usepackage{textcomp}

% \renewcommand{\encodingdefault}{LY1}
% \rmshape \rgshape

% \usepackage[oldstyle]{agaramond}
%\usepackage[lining]{agaramond}
% \usepackage[small]{eulervm}
% \usepackage{courier} % for texttt



% \usepackage{ulem} %underlines
% \normalem % normal emph w/ ulem

\usepackage[numbers,square,sort&compress,sectionbib]{natbib}
\usepackage{chapterbib}
\usepackage[pdftex,plainpages=false,breaklinks=true,colorlinks=true,urlcolor=blue,citecolor=blue, linkcolor=blue,bookmarks=true,bookmarksopen=true,bookmarksopenlevel=0,pdfstartview=Fit,pdfview=Fit,pagebackref,linktocpage=true,bookmarksnumbered=true]{hyperref}
\usepackage{hypernat}
\usepackage{array}
\usepackage{supertabular}
\usepackage{booktabs}
\usepackage{placeins}

%%% stuff for doxygen
%\usepackage{times}
\usepackage{multicol}
\usepackage{multirow}
\usepackage{float}
\usepackage{alltt}
\graphicspath{{Figures/}}
% \usepackage{Body/appb/doxygen}
%%%%%%%%%%%

\usepackage{fancyhdr}
%\renewcommand{\chaptermark}[1]{\markboth{\textit{\chaptername}\ \thechapter.\ #1}{}}

\newcommand\blfootnote[1]{%
  \begingroup
  \renewcommand\thefootnote{}\footnote{#1}%
  \addtocounter{footnote}{-1}%
  \endgroup
}

% Declare argmin as a math operator
\DeclareMathOperator*{\argmin}{arg\,min}

%this defines the basic headers and footer
% styles when we use the 'fancyhdr' styles
%\lhead[\fancyplain{}{\itshape\footnotesize\thepage}]{\fancyplain{}{\itshape\footnotesize\rightmark}}
%\rhead[\fancyplain{}{\itshape\footnotesize\leftmark}]{\fancyplain{}{\itshape\footnotesize\thepage}}
%\lhead[\fancyplain{}\bfseries\thepage]{\fancyplain{}\bfseries\rightmark}
%\rhead[\fancyplain{}\bfseries\leftmark]{\fancyplain{}\bfseries\thepage}
%\pagestyle{fancyplain}
\addtolength{\headwidth}{0.5\marginparsep}
\addtolength{\headwidth}{0.5\marginparwidth}
%\renewcommand{\chaptermark}[1]{\markboth{#1}{}}
%\renewcommand{\sectionmark}[1]{\markright{\thesection\ #1}}
\lhead[\fancyplain{}{\footnotesize\thepage}]{\fancyplain{}{\footnotesize\rightmark}}
\rhead[\fancyplain{}{\footnotesize\leftmark}]{\fancyplain{}{\footnotesize\thepage}}
\cfoot{}
\cfoot{}

% Special Float captions
% Different font in captions
\newcommand{\captionfonts}{\mdseries}
\newcommand{\floatnamefonts}{\bfseries}
\makeatletter  % Allow the use of @ in command names
\long\def\@makecaption#1#2{%
  \vskip\abovecaptionskip
  \sbox\@tempboxa{{\floatnamefonts #1:~~\captionfonts #2}}%
  \ifdim \wd\@tempboxa >\hsize
    {\floatnamefonts #1: \captionfonts #2\par}
  \else
    \hbox to\hsize{\hfil\box\@tempboxa\hfil}%
  \fi
  \vskip\belowcaptionskip}
\makeatother   % Cancel the effect of \makeatletter

\makeindex
