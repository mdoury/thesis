\chapter*{Foreword}
In the first part of this thesis, we reviewed the methodological developments of perfusion quantification methods.
Various modalities were historically used to assess perfusion, from the most invasive requiring catheterization for blood sampling to the most advanced in-vivo imaging techniques, however they all require injecting a tracer to monitor its concentration throughout the experiment.
Quantification of perfusion consists in the estimation of parameters characterizing the physiology of the tissue under investigation, in particular regarding the distribution of blood or the exchanges between blood and tissue.

Chapter~\ref{chapter:review} addresses the three main quantification approaches used to characterize perfusion exams, i.e.~semi-quantitative, deconvolution, and compartmental.
The semi-quantitative approaches are the most intuitive, they derive perfusion parameters directly from the tracer concentration curves, from which physiological parameters can be derived.
Deconvolution approaches consider the tissue as a black-box system fed by an arterial input, and estimate the tissue response to an instantaneous injection making few assumptions on the underlying physiology and tracer characteristics. 
Deconvolutions result in impulse responses with unknown shapes, from which perfusion parameters are usually derived.
Compartmental models can be viewed as explicit deconvolution, where the shape of the impulse response in known and parameterized by physiologically relevant parameters.

The methods are presented by imaging modalities, however methodological transfers between modalities was common and was emphasized when a clear continuity was observed. 
The evolution of a technique is presented chronologically when possible to reveal the incremental development of the methods.
\blfootnote{N.B.~All the references cited in this part of the thesis are pooled together and presented at the end of Part~\ref{part:part1}.}