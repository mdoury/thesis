\chapter*{Foreword}
In the second part of this thesis, we perform an in-depth investigation of three major perfusion quantification techniques.
The techniques investigated are a semi-quantitative method using the log-normal distribution, a one-compartment model using an arterial measurement, and a one-compartment model using a reference tissue. 
The first two methods yield absolute parameters while the third directly yields relative parameters.
Relative parameters were derived from the absolute parameters using their value inside a reference tissue.
In an attempt to reveal the spatial functional heterogeneity of tumors while ensuring reasonable signal-to-noise ratio in the time-intensity curves, the models were fitted to regional time-intensity curves resulting from the cutout of the perfused tumor area. 

Our group previously proposed a multiplicative noise model to reflect the characteristics of noise in ultrasound data.
The multiplicative noise model was not used in this study however, as it was unable to accurately fit the regional data despite using the least-square estimates as initialization values. 
Instead, we used a non-linear least-square estimation method to fit all models.

The log-normal model is commonly used in the litterature to fit perfusion curves, and in particular contrast-enhanced ultrasound data.
It is mainly used to filter out the noise of the time-intensity curves, and semi-quantitative parameters are usually derived from the fitted curve, e.g.~the area under the curve, the peak enhancement, or the wash-in rate. 
Our research group used the log-normal model to fit the global mean time-intensity curve inside the perfused area of tumors.
The author of this study revealed the poor reproducibility of the global semi-quantitative parameters despite the improvement resulting from the use of an automatic injection setup, which was also used in Chapter~\ref{chapter:PMB}.

Compartmental modeling performs a normalization of the studied time-intensity curve by the input of the system through deconvolution, theoretically making parameters independant on the characteristics of the tracer injection, i.e.~quantity, duration.
Compartmental approaches were primarily developed for nuclear medicine imaging techniques, they were then extended to other perfusion imaging modalities, in particular to contrast-enhanced magnetic resonance imaging.
The one-compartment model is the most basic form of compartmental model, and is particularly designed for intravascular tracers like ultrasound contrast agents. 
However no reference of this model applied to contrast-enhanced ultrasound data was found in the litterature.
The method allows the estimation of absolute perfusion parameters, i.e.~tissue blood flow, and tissue blood volume.
However, as discussed in Chapter~\ref{chapter:review}, arterial measurements are subject to artifacts, and are in fact not always possible.

To alleviate the need for arterial measurements, relative methods making use of a reference tissue instead were developed, allowing direct estimation of relative perfusion parameters.
These relative parameters are closely related to the absolute parameters, except they are normalized according to the reference tissue.
The tissue interest can be selected in a large and homogeneous region free of large blood vessels, reducing the influence of noise, saturation, and partial volumes.

The reproducibility of the regional parameters estimated using the two absolute methods and the three relative methods was investigated through test-retest experiments in order to limit the inter-exam changes to their minimum.
Additionally, the impact of accounting for the time taken by the tracer to arrive in the tissue on the reproducibility of the estimated perfusion parameter was studied. 
The experimental setup and the results are reported in Chapter~\ref{chapter:PMB} which is the content of an article published in {\em Physics in Medicine and Biology}.
In Chapter~\ref{chapter:PMB2}, we propose an extention of this work establishing the relations between the parameters of the three models mentioned aboove, first theoretically, and then experimentally through a correlation study.
The references cited in these two chapters are pooled together and presented at the end of this part of the thesis.