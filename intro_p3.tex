\chapter*{Foreword}
Chapter~\ref{chapter:PMB} revealed the superior reproducibility of the relative perfusion parameters estimated by the reference tissue model.
However, as discussed in the afterword of the second part of this thesis, the formulation of the reference tissue model used in this study requires a fixed value of parameter $k_R$,i.e.~the rate constant characterizing the time-intensity curve in the reference tissue.
In the third and last part of this thesis, we propose a linear formulation of the reference tissue model alleviating the need for a fixed value of $k_R$, and assess it using both experimental and simulated data.

Using the linear formulation for regional analysis, a value of $k_R$ should be estimated for each region.
Such discrepencies in the values of $k_R$ can affect the value of the other perfusion parameters.
This issue is further discussed in Chapter~\ref{chapter:IUS}, where a regularized estimation method ensuring a single value of $k_R$ is estimated per exam.
The reproducibility of the linear resolution method and of the proposed regularized method are assessed using the same test-retest experiments as presented in Chapter~\ref{chapter:PMB}, and compared to the absolute and normalized parameters of the Log-Normal model, i.e.~$AUC$, $rAUC$, $WIR$, $rWIR$.
This work was published in the proceedings of the {\em IEEE International Ultrasonics Symposium} (IUS)~\cite{Doury:2016fi}.

The reviewers of \cite{Doury:2016fi} raised an interesting question regarding the impact of recirculation on the perfusion parameters estimated in our contrast-enhanced ultrasound test-retest study.
Chapter~\ref{chapter:IRBM} address this issue in depth and study the impact of recirculating microbubbles on the accuracy and precision of the perfusion parameters through a simulation study.
A simple recirculation model was used, it was however able to reflect the various passes of the bolus, as well as its dispersion and attenuation.
An intuitive approach to avoid recirculation is to fit the model to the portion of data acquired before recirculation occurs, thus a classical and a cropped version of the Log-Normal model were assessed.
Additionally, the robustness of the linear reference tissue model, and of its regularized version to recirculation, was assessed.
This work has been submitted to {\em Innovation and Research in BioMedical Engineering} (IRBM).

Chapter~\ref{chapter:PLOSONE} formally explains the linear and regularized estimation methods to solve the reference tissue model.
Indeed, getting into details was not possible in~\cite{Doury2017wn} because of the limited space available in the proceedings template.
This is however not the core of this work, which aims at evaluating the sensitivity of perfusion parameters to varying data characteristics through simulations.
In particular, the impact of acquisition duration, acquisition frequency, noise amplitude, number of tissues, as well as the characteristics of the reference tissue were investigated.