\chapter*{Afterword}
The development of quantification methods often came along with the development of the imaging modality itself.
Semi-quantitative approaches are the most intuitive approaches, they are generally used to characterize perfusion data in the early stage of the imaging technique, they are however subject to inter-exam physiological and experimental changes.
Then physiological parameters are often derived from semi-quantitative parameters characterizing the tracer kinetic in the tissue of interest and in an artery, they however suffer from the same limitations as semi-quantitative parameters, but are also affected by the difficulty of estimating the arterial curve.

Ultrasound contrast agents are not anymore in their early development stages, on the contrary they reached a point where their behavior is well understood and where they can be used routinely for some clinical applications. 
While the ultrasound imaging technique is not new at all, it was using analog signal processing for a long period of time.
It however reached a turning point with the development of numerical ultrasound scanners, exploiting the development of high-end graphical hardware which allows parallel computing to allow complex real-time signal processing.
This turning point suggest future development of the imaging technique, in particular with the joint development of ultrafast plane wave imaging and three-dimensional ultrasound probes.

Regarding contrast-enhanced ultrasound, which is the core modality addressed in this thesis, semi-quantitative approaches are by far the most commonly used approaches.
Indeed, many manufacturers of ultrasound scanners and contrast-agent implemented these techniques in commercial softwares, which explains their popularity in clinical appication.
A few approaches relying on deconvolution were also proposed for characterization of contrast-enhanced ultrasound exams, but compartmental modeling remains extremely rare in the ulrasound litterature.
This can be explained by the dependance of the majority of these techniques on the ability to perform an accurate arterial measurement. 
However accurate measurement is not always possible as arterial regions are usually small and exhibit high tracer concentration, making the estimated curve subject to saturation and partial volume effects.
Finding an artery can be especially tricky in two-dimensional data, and even more limiting when attempting to compare two or more exams.
Indeed, imaging the exact same plane is extremely difficult even for experienced radiologists, especially in the case of evolving tissues like growing tumors or when monitoring the effect of a treatment.
The impact of the arterial function on the parameters of a one-compartment model for quantification of contrast-enhanced ultrasound data will be investigated in Chapter~\ref{chapter:PMB}.

When an arterial input function cannot be estimated, or at least not accurately, another tissue present in the image can be used as a reference for comparison purposes.
Just like the absolute values, the relative or normalized perfusion parameters can be used to perform relative comparison of the tissues observable in a single exam. 
However, parameter normalization allows comparison of the same tissue observed in different exams, using the reference tissue as a basis for comparison, and assuming the reference tissue did not change between the two exams.
The reproducibility of relative approaches will be investigated in the following part of the thesis.

