\chapter*{Afterword}
The analytical expression of the area under the curve reveals its straightforward relation with the fractional blood volume.
This semi-quantitative parameter is proportional to the physiological parameter, and can be used for relative comparison of tissues observed in a single exam.
Furthermore, using the normalized area under the curve successfully corrects for the inter-exam variations, and therefore makes the comparison more robust.

The relation of the wash-in rate with the tissue blood flow is more complex, and while the normalized parameter is analitically and empirically proportional to the normalized blood flow, the slope of the relation varies from one experiment to another.
This explains the weaker improvement observed in terms of reproducibility for the normalized wash-in rate compared to the absolute value.

This study reveals the sensitivity of the arterial measurement to segmentation.
Indeed, arterial regions exhibiting both small areas and high signal intensities, small changes in the segmentation can result in large changes in the mean arterial curve.
In addition, a log-normal model was fitted to the arterial curve for noise-filtering prior to quantification.
Given the noise level in those high intensity regions, the fitted curve is likely biased.
Absolute perfusion parameters were strongly affected by varyiation of the segmented artery region, however relative parameters exhibited a better agreement.

This formulation of the reference tissue model introduces an unidentifiable parameter, i.e.~the rate constant in the reference tissue.
In this study, this parameter was given the mean value obtained with the one-compartment model using the arterial measurement.
And while the estimated parameters are most likely biased depending on the discrepency of the fixed parameter value with its actual value, direct estimation of relative perfusion parameters was the most robust approach in our study.
In the next part of this thesis, a linear formulation of the reference tissue model is presented to address this issue, along with a regularized estimation scheme to improve parameter reproducibility and comparability.