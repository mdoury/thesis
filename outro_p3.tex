\chapter*{Afterword}
In the third part of this thesis, we proposed and assessed a new quantification model based on a one-compartment model using a reference tissue that allows simultaneous, robust, and homogeneous estimation of perfusion parameters in multiple regions.
The model ensures that a single value of the rate constant characterizing the reference tissue is estimated across the various regions of analysis through regularization of this parameter.

In Chapter~\ref{chapter:IUS} we first demonstrated the superiority of the approach in terms of parameter reproducibility through preclinical test-retest experiments.
Indeed, the regularized model yielded the most reproducible perfusion parameters among the investigated methods.
Then, the robustness of the model to recirculation of the contrast agent throughout the course of the acquisition was assessed on simulated data in Chapter~\ref{chapter:IRBM}.
This study confirmed the robustness of compartmental models to recirculation, and revealed the sensitivity of semi-quantitative parameters to this physiological phenomenon.
Finally, in Chapter~\ref{chapter:PLOSONE} we investigated the impact of acquisition characteristics, i.e.~noise level, exam duration, and sampling period, as well as analysis strategies, i.e.~number of regions, choice of reference tissue.
The regularized model proved robust to noise, and relaxed the requirements for temporal resolution and exam duration.
Additionally, regularizing the estimation of $k_R$ made the estimation more homogeneous across regions, allowing meaningful comparison of perfusion parameters between regions.
The choice of the reference tissue was proved to affect parameter estimation, and the role of parameter $k_R$ was revealed.
This first analysis pushes us to recommend the use of a large and well perfused tissue as a reference.
